\PassOptionsToPackage{unicode=true}{hyperref} % options for packages loaded elsewhere
\PassOptionsToPackage{hyphens}{url}
%
\documentclass[12pt,oneside]{book}
\usepackage{lmodern}
\usepackage{amssymb,amsmath}
\usepackage{ifxetex,ifluatex}
\usepackage{fixltx2e} % provides \textsubscript
\ifnum 0\ifxetex 1\fi\ifluatex 1\fi=0 % if pdftex
  \usepackage[T1]{fontenc}
  \usepackage[utf8]{inputenc}
  \usepackage{textcomp} % provides euro and other symbols
\else % if luatex or xelatex
  \usepackage{unicode-math}
  \defaultfontfeatures{Ligatures=TeX,Scale=MatchLowercase}
\fi
% use upquote if available, for straight quotes in verbatim environments
\IfFileExists{upquote.sty}{\usepackage{upquote}}{}
% use microtype if available
\IfFileExists{microtype.sty}{%
\usepackage[]{microtype}
\UseMicrotypeSet[protrusion]{basicmath} % disable protrusion for tt fonts
}{}
\IfFileExists{parskip.sty}{%
\usepackage{parskip}
}{% else
\setlength{\parindent}{0pt}
\setlength{\parskip}{6pt plus 2pt minus 1pt}
}
\usepackage{hyperref}
\hypersetup{
            pdfborder={0 0 0},
            breaklinks=true}
\urlstyle{same}  % don't use monospace font for urls
\usepackage{longtable,booktabs}
% Fix footnotes in tables (requires footnote package)
\IfFileExists{footnote.sty}{\usepackage{footnote}\makesavenoteenv{longtable}}{}
\usepackage{graphicx,grffile}
\makeatletter
\def\maxwidth{\ifdim\Gin@nat@width>\linewidth\linewidth\else\Gin@nat@width\fi}
\def\maxheight{\ifdim\Gin@nat@height>\textheight\textheight\else\Gin@nat@height\fi}
\makeatother
% Scale images if necessary, so that they will not overflow the page
% margins by default, and it is still possible to overwrite the defaults
% using explicit options in \includegraphics[width, height, ...]{}
\setkeys{Gin}{width=\maxwidth,height=\maxheight,keepaspectratio}
\setlength{\emergencystretch}{3em}  % prevent overfull lines
\providecommand{\tightlist}{%
  \setlength{\itemsep}{0pt}\setlength{\parskip}{0pt}}
\setcounter{secnumdepth}{5}

% set default figure placement to htbp
\makeatletter
\def\fps@figure{htbp}
\makeatother

% L D'Agostino McGowan dissertation preamble
% Adapted from L Samuels dissertation template
% Adapted from Qi Liu's template

%%%%%%%%%%%%%%%%%%%%%%%%%%%%%%%%%%%%%%%%%%%%%%%%%%%%%%%%%%
\newcommand{\myname}{Your full name}
\newcommand{\maintitle}{Your title}
\newcommand{\myuniversity}{Your University}
\newcommand{\mycitystate}{City, State of your University}
\newcommand{\mydiscipline}{Your Discipline}
\newcommand{\Chair}{Committee Chair, Ph.D.}
\newcommand{\Advisor}{Advisor, Ph.D.}
\newcommand{\OtherInternalMember}{Committee Member, Ph.D.}
\newcommand{\ExternalMember}{Committee member, Ph.D.}

\newcommand{\mygraddate}{Your graduation date}
\newcommand{\mypubyear}{Your publication year}
%%%%%%%%%%%%%%%%%%%%%%%%%%%%%%%%%%%%%%%%%%%%%%%%%%%%%%%%%%

\usepackage{alltt} % for knitr code output
\usepackage{etoolbox} % for toggle
\usepackage[left=1.5in,right=1in,top=1in,bottom=1in]{geometry}
\usepackage{nomencl} % for the nomenclature (List of Abbreviations)
\usepackage{amsmath, amsfonts, amssymb, amsthm}
\usepackage{amsbsy,bm} 
\usepackage{lmodern} % Latin modern fonts
\usepackage[T1]{fontenc} % font encodings
\usepackage{framed} % for knitr code output
\usepackage{graphicx}
\usepackage{multirow} % need this for table in Paper 1
\usepackage[font=singlespacing]{caption}
\usepackage{setspace}
\usepackage{titlesec}
\usepackage[table]{xcolor}
\usepackage{float}
\usepackage{colortbl} % allows coloring of individual cells
\usepackage{natbib}
\usepackage{mathtools} % allows split fractions
\usepackage{indentfirst} % Indent first paragraph at beginning of section
\usepackage{subfig} % This one is required by something else
\usepackage[titletoc,title]{appendix}
\usepackage[tocgraduated]{tocstyle} % This package is still in alpha, but we need it for VU
\usepackage{array}
\usepackage{listings} % Typeset source code listings using LATEX
\usepackage{makeidx}
\usepackage{booktabs}
\usepackage[subfigure, titles]{tocloft} % lets us format TOC for VU
\usepackage[nottoc,notlof,notlot]{tocbibind}
\usepackage{pdflscape}
\usepackage{placeins} % for \FloatBarrier
\hypersetup{
    bookmarksnumbered=false, % true means bookmarks in left window are numbered
    bookmarksopen=false,     % true means only level 1 is displayed.
    colorlinks=true,         % false: boxed links; true: colored links
    linkcolor=black,          % color of internal links
    citecolor=black,          
    urlcolor=black,          
    pdfstartview=FitH
}
\usepackage{fancyhdr}
\pagestyle{fancy}
\fancyhead{}% clear headers
\fancyfoot{}% clear footers
\renewcommand{\headrulewidth}{0pt}% eliminate horizontal line
\fancyfoot[c]{\thepage}

\urlstyle{same}
\bibliographystyle{vandy_dissertation}
\setcounter{secnumdepth}{7}
\setcounter{tocdepth}{7}
\captionsetup{font=footnotesize}
\setcitestyle{citesep={;}, aysep={,}}

\graphicspath{{paper1/figs/}{paper2/figs/}{paper3/figs/}}
\DeclareGraphicsExtensions{.pdf,.jpeg,.png,.PNG,.eps,.tiff}

\textwidth=6in \oddsidemargin=0.5in \topmargin=-0.5in
\textheight=9in  % 9in must include page numbers
\textfloatsep = 0.4in 

\addtocontents{toc}{\vspace{0.4in} \protect \hfill Page\endgraf} 
\addtocontents{lof}{\vspace{0.2in} \hspace{0.13in} \ Figure \protect \hfill Page\endgraf} 
\addtocontents{lot}{\vspace{0.2in} \hspace{0.13in} \ Table \protect\hfill Page\endgraf}

%%%%%%%%%%%%%%%%%%%%%%%%%%%%%%%%%%%%%%%%%%%%%%%%%%%%
%%%%%%%%%%%%%%%% Table of contents %%%%%%%%%%%%%%%%%
\renewcommand{\contentsname}{TABLE OF CONTENTS}
\renewcommand{\cftchapfont}{\normalfont}
\renewcommand{\cftchappagefont}{\normalfont}
\renewcommand{\cftchapleader}{\cftdotfill{\cftdotsep}}


%%%%%%%%%%%%%%%%%%%%%%%%%%%%%%%%%%%%%%%%%%%%%%%%%%%%
%%%%%%%%%%%%%%%%% List of figures %%%%%%%%%%%%%%%%%%
\renewcommand{\listfigurename}{LIST OF FIGURES}
\setlength{\cftbeforefigskip}{0.2in}

%%%%%%%%%%%%%%%%%%%%%%%%%%%%%%%%%%%%%%%%%%%%%%%%%%%%
%%%%%%%%%%%%%%%%% List of tables %%%%%%%%%%%%%%%%%%%
\renewcommand{\listtablename}{LIST OF TABLES}
\setlength{\cftbeforetabskip}{0.2in}

%%%%%%%Configure List of Abbreviations%%%%%%%%%%%%
\renewcommand{\nompreamble}{\vspace{0.25in}} % code after main title
\makenomenclature
\renewcommand{\nomname}{LIST OF ABBREVIATIONS}

%%%%%%%%%%%%%%%%%%%%%%%%%%%%%%%%%%%%%%%%%%%%%%%%%%%%
%%%%%%%%%%%%%%%%%% Bibliography %%%%%%%%%%%%%%%%%%%%
\renewcommand{\bibname}{\texorpdfstring{{REFERENCES\vspace{10mm}}}{REFERENCES}}

%%%%%%%%%%%%%%%%%%%%%%%%%%%%%%%%%%%%%%%%%%%%%%%%%%%%
%%%%%%%%%%%%%%%% Chapter headings %%%%%%%%%%%%%%%%%%
\makeatletter
\def\@makechapterhead#1{
  {\parindent \z@ \centering
    \normalfont
    \ifnum \c@secnumdepth >\m@ne
      \if@mainmatter
        \MakeUppercase{\@chapapp}\space \thechapter
        \par\nobreak
        \vskip 20\p@
      \fi
    \fi
    \interlinepenalty\@M
    #1\par\nobreak
    \vskip 40\p@
  }}

\def\@schapter#1{\if@twocolumn
                   \@topnewpage[\@makeschapterhead{#1}]%
                 \else
                   \@makeschapterhead{#1}%
                   \@afterheading
                 \fi}

\def\@makeschapterhead#1{
  {\parindent \z@ \centering
    \normalfont
    \interlinepenalty\@M
    #1\par\nobreak
    \vskip 10\p@
  }}


%%%%%%%%%%%%%%%%%%%%%%%%%%%%%%%%%%%%%%%%%%%%%%%%%%%%
%%%%%%%%%%%%%%%% TOC line spacing %%%%%%%%%%%%%%%%%%
\newlength{\li}\setlength{\li}{14.48pt}
\newlength{\di}\setlength{\di}{-3.5mm}

\def\@chapter[#1]#2{\ifnum \c@secnumdepth >\m@ne
      \refstepcounter{chapter}%
      \typeout{\@chapapp\space\thechapter.}%
      \addcontentsline{toc}{chapter}{\numberline{\thechapter}#1}
         %{\protect\numberline{\thechapter}\uppercase{#1}}%
      \addtocontents{toc}{\protect\vspace{\li}}%
  \else
      %\addcontentsline{toc}{chapter}{\uppercase{#1}}%
      \addcontentsline{toc}{chapter}{#1}
      \addtocontents{toc}{\protect\vspace{\li}}%
  \fi
  \chaptermark{#1}%
  \if@twocolumn
      \@topnewpage[\@makechapterhead{\MakeUppercase{#2}}]%
  \else
      \@makechapterhead{\MakeUppercase{#2}}%
      \@afterheading
 \fi}


\renewcommand\chapter{\addtocontents{toc}{\protect\addvspace{\li}}%
  \if@openright\cleardoublepage\else\clearpage\fi
  \thispagestyle{plain}%
  \global\@topnum\z@
  \@afterindentfalse
  \secdef\@chapter\@schapter}

\newdimen\@bls                              
\@bls=\baselineskip


%%%%%%%%%%%%%%%%%%%%%%%%%%%%%%%%%%%%%%%%%%%%%%%%%%%%
%%%%%%%%%%%%%%%% Section headings %%%%%%%%%%%%%%%%%%
\renewcommand\section{ \@startsection {section}{1}{\z@}%
    %{-2\@bls}%
    {2\@bls  plus .3\@bls minus .1\@bls}%
    {0.1\@bls}%
    {\centering\normalfont}}

%%%%%%%%%%%%%%%%%%%%%%%%%%%%%%%%%%%%%%%%%%%%%%%%%%%%
%%%%%%%%%%%%%% Sub-section headings %%%%%%%%%%%%%%%%
\renewcommand\subsection{\@startsection {subsection}{2}{\z@}%
    %{-2\@bls}%
    {2\@bls  plus .3\@bls minus .1\@bls}%
    {0.1\@bls}%
    {\noindent\normalfont}}

%%%%%%%%%%%%%%%%%%%%%%%%%%%%%%%%%%%%%%%%%%%%%%%%%%%%
%%%%%%%%%%%% Sub-sub-section headings %%%%%%%%%%%%%%
\renewcommand\subsubsection{\@startsection {subsubsection}{3}{\z@}%
    %{-1\@bls}%
    {\@bls plus .2\@bls}%
    %{-5pt}%
    {0.1\@bls}%
    %{\noindent\normalfont\bfseries}}
    {\noindent\normalfont\itshape}}

%%%%%%%%%%%%%%%%%%%%%%%%%%%%%%%%%%%%%%%%%%%%%%%%%%%%
%%%%%%%%%%%%%%%%%% TOC headings %%%%%%%%%%%%%%%%%%%
\renewcommand{\@cftmaketoctitle}{
  \chapter*{\contentsname}
  \addcontentsline{toc}{chapter}{TABLE OF CONTENTS}} 

%%%%%%%%%%%%%%%%%%%%%%%%%%%%%%%%%%%%%%%%%%%%%%%%%%%%
%%%%%%%%%%%% List of figures headings %%%%%%%%%%%%%%
\renewcommand{\@cftmakeloftitle}{
  \chapter*{\listfigurename}
  Figure \hfill Page
  \addcontentsline{toc}{chapter}{LIST OF FIGURES} } 
  
%%%%%%%%%%%%%%%%%%%%%%%%%%%%%%%%%%%%%%%%%%%%%%%%%%%%
%%%%%%%%%%%%% List of tables headings %%%%%%%%%%%%%%
\renewcommand{\@cftmakelottitle}{
  \chapter*{\listtablename}
   Table \hfill Page
   \addcontentsline{toc}{chapter}{LIST OF TABLES} }  

\makeatother
%%%%%%%%%%%%%%%%%%%%%%%%%%%%%%%%%%%%%%%%%%%%%%%%%%%
%%%%%%%%%%%%%%%%%%%%%%%%%%%%%%%%%%%%%%%%%%%%%%%%%%%

\setcounter{section}{-1}     

% Define floats here
\floatstyle{plain}
\newfloat{Box}{h}{lob}
\newcommand{\boxedtext}[3]{
    \begin{Box} \caption{\small{#1}}
    \hspace{1.cm}
    \fbox{\begin{minipage}[c]{0.85\linewidth} 
    
    \small{#2}
   
   \end{minipage}}
   
   \label{#3}
   \end{Box}
}

\let\mySpacing=\onehalfspacing

\date{}

\usepackage{amsthm}
\newtheorem{theorem}{Theorem}[chapter]
\newtheorem{lemma}{Lemma}[chapter]
\theoremstyle{definition}
\newtheorem{definition}{Definition}[chapter]
\newtheorem{corollary}{Corollary}[chapter]
\newtheorem{proposition}{Proposition}[chapter]
\theoremstyle{definition}
\newtheorem{example}{Example}[chapter]
\theoremstyle{definition}
\newtheorem{exercise}{Exercise}[chapter]
\theoremstyle{remark}
\newtheorem*{remark}{Remark}
\newtheorem*{solution}{Solution}
\begin{document}

\pagenumbering{alph}

%%%%%%%%%%%%%%%%%%%%%%%%%%%%%%%%%%%%%%%%%%%%%%%%%%%%
%%%%%%%%%%%%%%%%%%%% Title Page %%%%%%%%%%%%%%%%%%%%
\begin{titlepage}
\thispagestyle{empty}\enlargethispage{\the\footskip}%
\begin{center}
    {\setstretch{1.66} {\maintitle}\par }%
    \vskip.4in
    By
    \vskip .3in
    {\myname}
    \vskip .3in
    
    \begin{doublespace}
    Dissertation\\
    Submitted to the Faculty of the \\
    Graduate School of \myuniversity \\
    in partial fulfillment of the requirements \\
    for the degree of \\ [.1in]
    \end{doublespace}
    
    \MakeUppercase{DOCTOR OF PHILOSOPHY} \\[.1in]
    in \\[.1in]
    {\mydiscipline} \\[.25in]
    \mygraddate \\[.25in]
    \mycitystate
\end{center}
\vskip .5in

%%%%%%%%%%%%%%%%%%%%%%%%%%%%%%%%%%%%%%%%%%%%%%%%%%%%
%%%%%%%%%%%%%%%%%% Approved names %%%%%%%%%%%%%%%%%%
\begin{center}
\begin{doublespace}
Approved:\\
\Chair\\
\Advisor \\
\OtherInternalMember \\
\ExternalMember \\
\end{doublespace}
\end{center}
%%%%%%%%%%%%%%%%%%%%%%%%%%%%%%%%%%%%%%%%%%%%%%%%%%%%

%%%%%%%%%%%%%%%%%%%%%%%%%%%%%%%%%%%%%%%%%%%%%%%%%%%%
%%%%%%%%%%%% Uncomment for Signatures %%%%%%%%%%%%%%
%Approved: \hskip 2.9in Date:\\[1.2em]
% \hskip -0.24in Approved: \hskip 2.85in Date:\\[1.2em]
% \rule{3.5in}{.5pt} \hskip 0.1in \rule{2in}{.5pt} \\[.01in]
% \Chair \\[.14in]
% \rule{3.5in}{.5pt} \hskip 0.1in \rule{2in}{.5pt}  \\[.01in]
% \Advisor \\[.14in]
% \rule{3.5in}{.5pt} \hskip 0.1in \rule{2in}{.5pt} \\[.01in]
% \OtherInternalMember \\[.14in]
% \rule{3.5in}{.5pt} \hskip 0.1in \rule{2in}{.5pt} \\[.01in]
% \ExternalMember \\[.14in]
% \\[.14in]
%%%%%%%%%%%%%%%%%%%%%%%%%%%%%%%%%%%%%%%%%%%%%%%%%%%%
\end{titlepage}
 
%%%%%%%%%%%%%%%%%%%%%%%%%%%%%%%%%%%%%%%%%%%%%%%%%%%%
%%%%%%%%%%%%%%%%%% Copyright page %%%%%%%%%%%%%%%%%%
\thispagestyle{empty}%\enlargethispage{\the\footskip}%
\vspace*{\fill}
\begin{center}
Copyright \copyright ~\mypubyear~by \myname\\
All Rights Reserved
\end{center}
\vspace*{\fill}
\clearpage


\mySpacing
\pagenumbering{roman} \setcounter{page}{3}


%%%%%%%%%%%%%%%%%%%%%%%%%%%%%%%%%%%%%%%%%%%%%%%%%%%%
%%%%%%%%%%%%%%%%% Dedication page %%%%%%%%%%%%%%%%%%

\chapter*{}
\addcontentsline{toc}{chapter}{DEDICATION}
\vspace{7mm}
\vspace*{\fill}
\begin{center}
To the entire Chipman family: My parents, siblings and their families, and fianc\'{e}e Danielle.  

They matter most in my life.
\end{center}
\vspace*{\fill}

%%%%%%%%%%%%%%%%%%%%%%%%%%%%%%%%%%%%%%%%%%%%%%%%%%%%
%%%%%%%%%%%%%% Acknowledgements page %%%%%%%%%%%%%%%

\chapter*{ACKNOWLEDGEMENTS}
\label{ch:Acknowledgements}
\addcontentsline{toc}{chapter}{ACKNOWLEDGEMENTS}
\vspace{7mm}
I acknowledge that today is gonna be a good day.


%%%%%%%%%%%%%%%%%%%%%%%%%%%%%%%%%%%%%%%%%%%%%%%%%%%%
%%%%%%%%%%%%%%%% Table of contents %%%%%%%%%%%%%%%%%

\begin{singlespace}
\tableofcontents
\end{singlespace}

\clearpage
\addcontentsline{toc}{chapter}{\listtablename}
\listoftables

\clearpage
\addcontentsline{toc}{chapter}{\listfigurename}
\listoffigures

\clearpage
\addcontentsline{toc}{chapter}{\nomname}
\printnomenclature

\clearpage
\normalsize
\mySpacing
\pagenumbering{arabic}
\setcounter{page}{1}

\cftlocalchange{toc}{450pt}{0cm} %see tocloft documentation
\cftaddtitleline{toc}{chapter}{Chapter}{}
\cftlocalchange{toc}{1.55em}{2.55em}

\hypertarget{introduction}{%
\chapter{Introduction}\label{introduction}}

Here is the introduction to my thesis.

\hypertarget{rematching-on-the-fly-sequential-matched-randomization-and-a-case-for-covariate-adjusted-randomization}{%
\chapter{Rematching on-the-fly: Sequential Matched Randomization and a
case for covariate-adjusted
randomization}\label{rematching-on-the-fly-sequential-matched-randomization-and-a-case-for-covariate-adjusted-randomization}}

\hypertarget{introduction-1}{%
\section{Introduction}\label{introduction-1}}

Randomization eliminates systematic confounding and has been considered
the ``gold standard'' for clinical trials since World War II(Bothwell et
al. \protect\hyperlink{ref-Bothwell:2016bc}{2016}). And yet, the most
common approaches to randomization --- Block and Stratified
Randomization --- are just as old (Peirce and Jastrow
\protect\hyperlink{ref-Peirce:1884wra}{1884}; Hill
\protect\hyperlink{ref-Hill:1952hc}{1952}; Armitage
\protect\hyperlink{ref-Armitage:1982ji}{1982}). They are limited in
their ability to control the balance of baseline covariates.
Covariate-adjusted randomization, which includes Stratified Block
Randomization, eliminates from the randomization space singular
randomizations with poor baseline covariate balance and consequently
reduces the uncertainty in permutation-based inference. It also
increases the efficiency of model-based inference by orthogonalizing
baseline covariates with treatment assignment.

Balance is an important aim for clinical trials. Poor chance imbalances
on key baseline covariates can bring to question the face-validity of a
randomized trial(Rosenberger and Sverdlov
\protect\hyperlink{ref-Rosenberger:2008cm}{2008}; Leyland-Jones
\protect\hyperlink{ref-LeylandJones:2003kt}{2003}) especially when
estimating the Population Average Treatment Effect from the Sample
Average Treatment Effect. Also, greater covariate balance allows
trialists to estimate heterogenous treatment effect across key
subgroups. Trials with secondary aims for personalizing medicine to key
subgroups rely upon overall covariate balance(Diener-West et al.
\protect\hyperlink{ref-DienerWest:1989uq}{1989}; Fu, Zhou, and Faries
\protect\hyperlink{ref-Fu:2016jy}{2016}). To reduce chance imbalances,
the trialist may turn to covariate-adjusted randomization (the focus of
this paper) and / or fit a model that adjusts for chance
imbalances(Berchialla, Gregori, and Baldi
\protect\hyperlink{ref-Berchialla:2018bk}{2018}). With model-based
inference, the trialist must convince the clinical community their model
is fully transparent and sufficiently adequate(D. A. Freedman
\protect\hyperlink{ref-Freedman:2008eq}{2008}\protect\hyperlink{ref-Freedman:2008eq}{a},
\protect\hyperlink{ref-Freedman:2008em}{2008}\protect\hyperlink{ref-Freedman:2008em}{b};
Lin \protect\hyperlink{ref-Lin:2013jh}{2013}).

In non-sequential trials, where all patients or clusters are known
before randomization, covariate-adjusted randomization randomizes within
optimal partitions of the patients. Common and novel methods (and the
imbalance measure they optimize) include: Stratified Block Randomization
(exact covariate matches)(Fisher
\protect\hyperlink{ref-fisher1935design}{1935}), Matched Randomization
(overall sum of paired distances from a distance matrix)(Greevy et al.
\protect\hyperlink{ref-Greevy:2004ke}{2004}), Rerandomization (any
imbalance measure though commonly overall difference in covariate
distances)(Morgan and Rubin
\protect\hyperlink{ref-Morgan:2012iq}{2012}), Propensity-Constrained
Randomization (overall difference in propensity score variability)(Loux
\protect\hyperlink{ref-Loux:2014bu}{2014}), and Kernel Randomization (a
linear or non-linear function of covariate differences) (Kallus and 2018
\protect\hyperlink{ref-Kallus:2018um}{2018}). Stratified Block
Randomization randomizes within exact matches of categorized baseline
covariate levels whereas the other methods randomize within near matches
of continuous and categorical baseline covariates. Exact matching is
frequently limited in the number of matching baseline covariate levels
before overwhelming the sample size. Though still under development,
some theoretical and empirical evidence points to Kernel Randomization
as the preferred method for greatest balance(Kallus and 2018
\protect\hyperlink{ref-Kallus:2018um}{2018}).

With sequential enrollment, the full covariate pattern and optimal
partitions are unknown until the end of the study. Many of the
non-sequential methods have extensions for sequential enrollment yet
lose some optimization compared to non-sequential randomization.
Minimization handles the sequential optimization problem by directly
allocating treatments to achieve optimal balance on specified imbalance
criteria(Taves \protect\hyperlink{ref-Taves:1974hn}{1974}).
Randomization occurs when all treatment would have the same impact on
minimizing the imbalance criteria. Without randomization, Minimization
schemes may remain subject to systematic confounding. To relax
deterministic allocation, treatment may instead be allocated with a
biased coin favoring the optimal treatment per the imbalance criteria
(Pocock and Simon \protect\hyperlink{ref-Pocock:1975wd}{1975}; Atkinson
\protect\hyperlink{ref-Atkinson:1982kt}{1982}). A common
covariate-adjusted biased coin method includes Urn Randomization(Wei and
Lachin \protect\hyperlink{ref-Wei:1988if}{1988}). Of the remaining
non-sequential randomization schemes, Sequential Matched Randomization
and Sequential Rerandomization set a tuning parameter to sequentially
optimize balance according to their imbalance measure(Kapelner and
Krieger \protect\hyperlink{ref-Kapelner:2014cu}{2014}; Zhou et al.
\protect\hyperlink{ref-Quan:SeqRerand}{2018}).

This work achieves two purposes. First, it extends Sequential Matched
Randomization to recover some of the optimal balance achieved in single
batch Matched Randomization (when all patients or clusters are known
prior to randomization). The extensions allow (1) matches to rematch if
a better mate enters the study, (2) the tuning parameter for optimal
matches to adjust dynamically according to the number of current matches
and covariate distribution, and (3) patients to enroll in blocks.
Second, through a two-arm case study (n = 512 patients) it re-emphasizes
the value of sequential covariate-adjusted methods, as a whole, compared
to Block Randomization. Sequential allocation methods considered include
Block Randomization, Stratified Block Randomization, Urn Randomization,
Sequential Matched Randomization, Atkinson's Minimization with and
without a biased coin, and Begg and Iglewicz's Minimization. We do not
include Sequential Rerandomization as it requires determining an optimal
threshold.

\hypertarget{notation}{%
\section{Notation}\label{notation}}

We denote a study as having enrolled \(i = 1, \dots,\) to \(N\) patients
throughout \(b = 1,\dots,\) to \(M\) batches of sequential enrollment.
At the \(b^{th}\) batch of enrolling patients, denote the set of
unmatched patients as \(U_b\) and the number of patients in the set as
\(||U_b||\). Let \(||R_b||\) be the number of expected remaining study
entrants. In general, covariate-adjusted randomizations are adjusted to
\(p\) baseline covariates in which a categorical covariate having \(q\)
levels is coded by \(q-1\) dummy variables.

In the simulation section, we generate \(j = 1, \dots,\) to 20,000
randomization schedules for each allocation scheme.

\hypertarget{sequential-matching-and-sequential-rematching}{%
\section{Sequential Matching and Sequential
Rematching}\label{sequential-matching-and-sequential-rematching}}

Sequential Matched Randomization uses the matching on-the-fly algorithm
published elsewhere(Kapelner and Krieger
\protect\hyperlink{ref-Kapelner:2014cu}{2014}) yet is worth summarizing
to then extend. When the first set of patients individually enroll, they
are randomized to either treatment or control and form a ``reservoir''
of patients who have been randomized but, as of yet, have not matched
with any other patient. Once a reservoir of pre-specified size has built
up, subsequent enrolling patients are compared to reservoir members on
baseline covariates using a distance matrix. The entering patient
becomes mates with their best reservoir match if they meet a
pre-specified distance threshold of similarity, and the entering patient
receives the opposite treatment randomized to their mate. Both are
excluded from the reservoir of potential mates. If, however, the best
reservoir match is not similar enough, the entering patient is
randomized and joins the reservoir awaiting a mate. By the end of the
study, some patients may not have found a mate, and treatment allocation
may be imbalanced. Hereafter, we'll refer to the matching-on-the-fly
algorithm as Sequential Matching.

Common practice uses Mahalanobis Distance and builds the initial
reservoir to p+2 patients. This distance matrix reduces to euclidean
distances when covariates are independent, and p+2 observations are
required before distances may be uniquely estimated. Current literature
suggests setting the pre-specified threshold as a quantile from the F(p,
i - p) distribution(Kapelner and Krieger
\protect\hyperlink{ref-Kapelner:2014cu}{2014}); justification lies in
the fact that distances of normal covariates, when appropriately scaled,
asymptotically follow the F(p, i-p) distribution.

Throughout Sequential Matching, a patient may match before their better
matching mate enrolls. We allow mates to break and rematch so long as
patients do not match within their treatment group. For example, in a
two arm study, controls are not allowed to match with controls and
treatments to treatments. We refer to this re-matching on-the-fly
algorithm as Sequential Rematching and the randomization scheme as
Sequential Rematched Randomization.

\hypertarget{dynamic-and-empirical-threshold}{%
\section{Dynamic and Empirical
Threshold}\label{dynamic-and-empirical-threshold}}

Sequential Matching, as is, uses a fixed quantile of the F-distribution;
we formalize the use of a dynamic threshold from an
empirically-estimated distribution of randomly matched distances.
Lacking omniscience, an improperly selected fixed similarity threshold
may be problematic. An overly strict fixed threshold would yield no
matches and in turn may be no better than Complete Randomization. In
contrast, an overly relaxed threshold would degenerate to a block-two
randomization scheme vulnerable to subversion bias. Our proposed dynamic
threshold reflects the chance of matching an existing reservoir member
out of all potential mates including those yet to enroll. More formally,
this is the proportion \(Q_b\) where

\[
Q_b = \frac{||  U_b || - 1}{ ||U_b|| + ||R_b|| - 1}.
\] Although the number of possible mates in Rematching is the whole set
of entrants, we develop \(Q_b\) for use with both Sequential Matching
and Sequential Rematching.

Since not all covariates are normally distributed, we empirically
estimate a reference null distribution, \(F\), as suggested in the
matching on-the-fly paper(Kapelner and Krieger
\protect\hyperlink{ref-Kapelner:2014cu}{2014}). To estimate \(F\) at the
\(b^{th}\) batch of enrolling patients (\(F_b\)), a random set of
\(i/2\) matches are bootstrap sampled from the upper- (or lower-)
triangle of the distance matrix. The dynamic threshold is the averaged
\(Q_b\) percentile across bootstrap samples of \(F_b\). To achieve equal
treatment allocation, the threshold for matches is removed when the
reservoir size is the same or less than the number of patients left to
enroll.

\[
Threshold_b = \begin{cases} \hat{F}_b^{-1} (Q_b) &  || U_b || < || R_b || \\
                    \text{best match(es)}      &  || U_b || \ge || R_b || \\
      \end{cases}
\]

The threshold may also be removed to keep the reservoir size within a
specified maximum tolerated imbalance (Berger, Ivanova, and Deloria
Knoll \protect\hyperlink{ref-Berger:2003im}{2003}).

\hypertarget{batch-entry}{%
\section{Batch Entry}\label{batch-entry}}

As is, Sequential Matching requires matching patients one at a time.
However, trials will allocate treatments in batches of various
enrollment sizes. As when all patients are known at the study outset,
batch enrollment increases the chance of finding a better mate. We
extend Sequential Matching to allow for batches of multiple patients
using an ``optimal'' algorithm; mates are collectively determined based
upon the set of matches that yield the smallest sum of distances. The R
package nbpMatching easily finds optimal mates.

\hypertarget{case-study-reach-trial}{%
\section{Case Study: REACH Trial}\label{case-study-reach-trial}}

\hypertarget{context-data-preparation-and-simulation-set-up}{%
\subsection{Context, Data Preparation, and Simulation Set
up}\label{context-data-preparation-and-simulation-set-up}}

The Rapid Education/Encouragement And Communications for Health (REACH)
randomized clinical trial(Nelson et al.
\protect\hyperlink{ref-Nelson:2018bw}{2018}) provides text
message-delivered diabetes support for 12 months to help diabetic
patients manage glycemic control (as measured by Hemoglobin A1c) and
adhere to treatment medication. Patients are randomized into one of
three treatment arms with a 2:1:1 allocation ratio: no text message,
text message, text message and monthly phone coaching. Now with complete
enrollment, 512 patients are currently being followed up through 3, 6,
12, and 15 months.

At baseline, clinical and demographics covariates were collected that
may be associated with 12 month Hemoglobin A1c, including baseline
Hemoglobin A1c, age at enrollment, gender, years of education, years of
diabetes, race / ethnicity, medication type, income, and type of
insurance. Refer to supplement for the overall distribution of each
baseline covariate.

Randomization often occurred before all baseline covariates could be
collected. Most notably, baseline Hemoglobin A1c generally took
additional time to process. In this study, clinical coordinators
enrolled patients and sent baseline data to the Data Coordinating Center
for weekly batch randomization. Though REACH follow-up continues, all
baseline covariates have been obtained, and 418 patients have recorded
three month Hemoglobin A1c.

\hypertarget{simulations}{%
\subsection{Simulations}\label{simulations}}

With complete baseline covariates, we simulated the ability of various
treatment allocation schemes to achieve balance among baseline
covariates among treatment groups and compared the efficiency in
estimating predicted three month Hemoglobin A1c. Contender allocation
schemes focused on Block Randomization, Stratified Block Randomization,
Urn\((0,\beta)\) Randomization, single batch Matched Randomization,
Sequential Matched Randomization without the proposed extensions
(similar to current literature), Sequential Matched Randomization with
the dynamic threshold, Atkinson's Minimization Algorithm with a 2/3
biased coin (Efron 1971) and with deterministic allocation, and Begg and
Iglewicz Minimization. The minimization schemes aim to reduce the
standard error in estimating the treatment effect.

We carried out Block and Stratified Block Randomization using a block
size of two. In practice, blocks of size two are ill-advised for
increasing the risk of subversion(Berger, Ivanova, and Deloria Knoll
\protect\hyperlink{ref-Berger:2003im}{2003}) but provides in simulation
the greatest chance of equal treatment allocation especially when
stratifying with many levels. As a point of reference we included single
batch Matched Randomization as the optimal matching when all patients
are known prior to randomization; and, to compare against current
Sequential Matched Randomization literature, we included Sequential
Matched Randomization with a fixed \(20^{th}\) percentile threshold of
the F(p, i-p) distribution. Atkinson's Algorithm determines the most
impactful treatment for reducing the estimated treatment variability of
a pre-specified model, which we specified as conditioning on all
categorical covariates and first through third degree polynomials of
continuous covariates. When there were fewer observations than model
degrees of freedom, we determined the optimal treatment for the
pre-specified model using the generalized inverse(Senn, Anisimov, and
Fedorov \protect\hyperlink{ref-Senn:2010bg}{2010}).

For a realistic comparison to Stratified Block Randomization, each
covariate-adjusted allocation was adjusted to site and the most
predictive baseline covariates of three month Hemoglobin A1c ---
baseline Hemoglobin A1c, medication type, and time since diabetes
diagnosis (see supplement for how these were determined). And, though
impractical for Stratified Block Randomization, each covariate-adjusted
allocation scheme conditioned on all baseline covariates. Stratified
Randomization, Urn(0,\(\beta\)), and Begg and Iglewicz Minimization
require categorized covariates; for these schemes we conditioned on
categorized derivations of baseline Hemoglobin A1c \((<7, 7-8, 8+)\),
age \((\leq 60, >60)\), years of education \((\leq 12, >12)\), and time
since diabetes diagnosis \((<10, 10+)\).

Though REACH includes three treatment arms, we simplified to two equally
allocated arms --- receiving no or any text-message intervention.
Missing baseline covariates and three month Hemoglobin A1c were multiply
imputed using predictive meaning matching via the aregImpute function in
the R rms package; and single mean and mode imputation from the multiple
imputations was carried out to obtain a single complete dataset. Twenty
thousand treatment allocation schedules for each scheme were simulated
on the single mean imputed dataset. For each scheme's generated 20,000
treatment schedules, we summarize with boxplots the maximum (worst-case)
and average absolute Standardized Mean Difference of all baseline
covariates. We also summarize end-of-study treatment allocation
difference.

To compare the efficiency of different schemes, we generated outcomes
assuming a potential outcomes framework. Twenty thousand datasets were
created where a patient's outcome under the control arm equaled
predicted three month Hemoglobin A1c plus a random residual. Under the
REACH intervention arm, the outcome decreased by 0.5 (a beneficial
decrease in this population). By using predicted three month Hemoglobin
A1c, we fixed the effect of baseline covariates. The only random
components in this framework are the treatment assignment and a random
residual of predicted three month Hemoglobin A1c.

For the j\(^{th}\) generated potential outcomes dataset, we obtained an
observed study for each allocation scheme using the scheme's j\(^{th}\)
generated allocation schedule. From each observed study, we calculated
the permutation-based 95\% Confidence Interval Width of the Sample
Average Treatment Effect (difference in observed means) and model-based
95\% Confidence Interval Width of the estimated treatment effect. The
pre-specified Ordinary Least Square Model adjusted for all baseline
covariates with restricted cubic splines on each continuous covariate
and accounted for multiply imputed baseline covariates. On the
j\(^{th}\) potential outcomes dataset, we also calculated the relative
width of each scheme's Confidence Interval Width compared to the width
from Block Randomization. Block Randomization was carried out twice for
each generated dataset to compare one instance of Block Randomization to
another. From this we calculated the difference in effective sample size
of each scheme relative to Block Randomization.

In the supplement, we further calculate the average size of the
reservoir throughout enrollment among matched randomization methods and
the expected amount of randomization occurring under each minimization
scheme.

\hypertarget{simulation-results}{%
\subsection{Simulation Results}\label{simulation-results}}

By themselves, the set of most predictive covariates, including site,
are highly predictive of three-month Hemoglobin A1c (R\(^2\) = 0.43);
adjusting for all baseline covariates is slightly more predictive
(R\(^2\) = 0.46). Atkinson's Minimization only randomized the first
patient and is therefore excluded from comparisons regarding
permutation-based inference. For this reason, it is also difficult to
draw conclusions regarding its average performance balancing baseline
covariates.

Balance of Covariates: As a point of reference and across simulations,
Block Randomization's median worst-balanced baseline covariate had an
absolute Standardized Mean Difference of 0.21 (95\% Percentile
Confidence Interval: (0.13, 0.32); Figure 1) and average absolute
Standardized Mean Difference of 0.09 (0.06, 0.14) (Figure 2).

Adjusting randomization to the most predictive baseline covariates
yielded only trivial gain in lessoning the worst-case imbalance (maximum
absolute Standardized Mean Difference; Figure 1). Begg and Iglewicz
Minimization performed best in reducing the worst-case imbalance to 0.20
(0.12, 0.30). Income level and race / ethnicity remained difficult
covariates to balance (supplement).

Worst-case imbalance was better controlled when adjusting to all
baseline covariates (Figure 1). Sequenital Matched Randomization,
utilizing a dynamic and empirical threshold, reduced the absolute
Standardized Mean Difference to 0.14 (0.09, 0.22). This slightly
improved upon Sequential Matched Randomization without these extensions
0.16 (0.10, 0.25). Sequential Rematched Randomization had a worst-case
imbalance of 0.11 (0.07, 0.20), which was an improvement in recovering
the performance of a single batched Matched Randomization (0.09 (0.05,
0.16)). Begg and Iglewicz Minimization performed best among the
sequential allocation methods (0.10 (0.05, 0.18)).

Overall baseline covariate imbalance (average absolute Standardized Mean
Difference) improved similarly and non-trivially across allocation
schemes when adjusting randomization to most predictive baseline
covariates (Figure 2). Again, greater improvement occurred when
adjusting to all baseline covariates. Adding the dynamic and empirical
threshold improved the performance of Sequential Matched Randomization,
and Sequential Rematched Randomization recovered much of the optimal
performance of single batch Matched Randomization. Begg and Iglewicz
performed essentially as well as single batch Matched Randomization.

Improvement Upon Precision: Block Randomization achieved a treatment
effect permutation-based 95\% Confidence Interval width of 0.54 (0.50,
0.58) (Figure 3) and a fully-adjusted (all baseline covariates) ordinary
least squares 95\% Confidence Interval width of 0.38 (0.37, 0.39)
(Figure 4). Relative to itself, the 95\% Percentile Confidence Interval
of change in effective sample size was +/- 20 (Table 1).

The greatest precision gains to permutation-based inference came from
schemes adjusting for only the most predictive baseline covariates
(Figure 3). An exception was Begg and Iglewicz Minimization, which
performed essentially the same on median when adjusting for the most
predictive versus all baseline covariates.

Excepting Urn Randomization, all other schemes that adjusted allocation
to the most predictive covariates achieved a Confidence Interval width
of 0.43 or smaller, with single batch Matched Randomization achieving on
median the same efficiency as the model-based efficiency under Block
Randomization (Figure 3). Sequential Matched Randomization with the
dynamic and empirical threshold yielded a permutation-based Confidence
Interval width of 0.41 (0.38, 0.44), an improvement on Sequential
Matched Randomization without extensions 0.43 (0.40, 0.47). Sequential
Rematched Randomization (0.40 (0.36, 0.45)) further recovered some of
the efficiency lost from single batch Matched Randomization (0.38 (0.36,
0.41)). On median, Atkinson's Algorithm with a Biased Coin yielded the
greatest permutation-based precision, nearly doubling the effective
sample size compared to Block Randomization. That is, Block
Randomization would have needed on median 436 (316, 572) additional
patients to achieve the same amount of precision. Sequential Rematched
Randomization increased the effective sample size by 416 (213, 663).

When adjusting randomization to all baseline covariates, efficiency
gains in permutation-based inference were not as pronounced but still
substantial (Figure 3). Sequential Rematched Randomization achieved a
permutation-based Confidence Interval Width of 0.46 (0.41, 0.51).
Atkinson's Minimization with a Biased Coin and Begg and Iglewicz
Minimization achieved the greatest permutation-based Confidence Interval
Width precision, 0.41 (0.40, 0.42) and 0.42 (0.38, 0.49) respectively.
On median, these were both superior to single batch Matched
Randomization.

In contrast to permutation-based inference, schemes that adjusted to all
baseline covariates achieved the greatest gain in precision from a
linear model adjusting to all baseline covariates (Figure 4). While
non-trivial, the gains relative to Block Randomization were much less
pronounced.

When adjusting to all baseline covariates, the best performing
sequential enrollment schemes included Sequential Rematched
Randomization, Atkinson's Algorithm with and without a Biased Coin, and
Begg and Iglewicz Minimization with Confidence Interval widths of at
most, on median, 0.376. Sequential Rematched Randomization yielded an
effective increase in sample size of 15 (-4, 37) above and beyond the
pre-specified model-based inference under Block Randomization.

End-of-Study Allocation Differences: Block Randomization, Sequential
Matched Randomization with a dynamic and empirical thresold, Sequential
Rematched Randomization, and single batch Matched Randomization always
achieved equal end-of-study allocation (Figure 5). Begg and Iglewicz
Minimization was close to achieving equal end-of-study allocation.
Atkinson's Algorithm with a Biased Coin, which performed well in
achieving balance and efficiency ran a greater risk of ending the study
with a treatment imbalance 0.00 (-20.00, 20.00).

Randomization within Minimization: When adjusting Begg and Iglewicz
Minimzation to the most predictive covariates, randomization occurred
for roughly 20\% of patients (7.8\% of patients when adjusting to all
baseline covariates). See supplement.

Reservoir Size: See supplement for the average reservoir size for
Sequential Matched and Rematched Randomization.

\hypertarget{conclusions}{%
\section{Conclusions}\label{conclusions}}

We've introduced extensions to Sequential Matched Randomization that
recover much of the optimality lost from single batch Matched
Randomization due to sequential entry of patients. These extensions
include Sequential Rematching with a dynamic and empirical threshold and
the ability to randomize patients in enrollment blocks. Further, our
case study re-emphasizes (Ciolino et al.
\protect\hyperlink{ref-Ciolino:2011ff}{2011}) the value of
covariate-adjusted randomization for increasing overall covariate
balance and efficiency estimating the Population Average Treatment
Effect. The precision of the permutation-based Sample Average Treatment
Effect estimator achieved nearly the same efficiency as model-based
inference estimating the treatment effect.

Our work is consistent with other findings that there is a balance /
efficiency trade-off in choosing which covariates to adjust
randomization(Kallus and 2018
\protect\hyperlink{ref-Kallus:2018um}{2018}; {\textbf{???}}). Arguments
can be made to prioritize balance, especially when investigating
important subgroups and when the strength of relationship between
baseline covariates and the outcome is unknown. And, arguments are made
for prioritizing efficiency especially when chance imbalances may be
adjusted with a model (Medicine and 1999
\protect\hyperlink{ref-Atkinson:1999hq}{1999}). Weighted distance
measures can allow an investigator to design the study prioritizing
certain covariates over others(Greevy Jr. et al.
\protect\hyperlink{ref-GreevyJr:2012hp}{2012}).

In introducing Sequential Rematching, we add to a class of look-back
allocation schemes. Though patients have matched, their baseline
covariates may yet be updated or used again to increase covariate
balance and precision. Also in this class are Minimization schemes with
and without a biased coin and Sequential Rerandomization. Such schemes
are beneficial when some baseline covariates are not initially available
at the time of randomization as was the case with Baseline Hemoglobin
A1c in the REACH trial.

In this case study, Begg and Iglewicz Minimization performed frequently
among the top performing Sequenital allocation schemes and provided
equal randomization to, on average, roughly 20\% of patients (7\% when
adjust randomization to all baseline covariates). This provides some
reassurance against systematic confounding. Minimization is powerful
from a study design perspective, and randomization is critical for
eliminating systematic confounding. This work begs the question, what is
the extent of randomization necessary to sufficiently reduce systematic
confounding?

Many studies set sample size based on the average performance of block
randomization (for example by basing of a t-test). This case study
provides a cautionary reminder that half of such studies will have a
decreased effective sample size. Though all assumptions and operations
of a study may work perfectly, a non-adaptive trial may fail to find a
treatment effect due to chance alone. Covariate-adjusted randomization
helps protect against poor covariate imbalances and increases the
chances of a sufficient effective sample size all else the same.

\begin{figure}
\centering
\includegraphics{~/Dropbox/statistics/work/vandy/RandomizationMethods/journal-clinical trials/latex/MaxSMD.jpg}
\caption{\label{fig:MaxSMD}Boxplot of the maximum absolute Standardized Mean
Difference among all baseline covariates for each of 20,000 simulated
allocation sequences per allocation scheme. Matched-based Randomization
schemes are shaded in blue. Seq Matched (20p) is Sequential Matching
before applying proposed extensions; it uses a fixed 20th percentile of
the F(p, i-p) distribution. Min SE schemes are Minimization schemes
proposed by Atkinson and Begg and Iglewicz to reduce the standard error
of a pre-specified Ordinary Least Squares Model. Min SE (Atk, 2/3 Prob)
uses a 2/3 biased coin to randomize to the favorable arm under
Atkinson's Minimization.}
\end{figure}

\begin{figure}
\centering
\includegraphics{~/Dropbox/statistics/work/vandy/RandomizationMethods/journal-clinical trials/latex/AveSMD.jpg}
\caption{\label{fig:AveSMD}Boxplot of the average absolute Standardized Mean
Difference among all baseline covariates for each of 20,000 simulated
allocation sequences per allocation scheme.}
\end{figure}

\begin{figure}
\centering
\includegraphics{~/Dropbox/statistics/work/vandy/RandomizationMethods/journal-clinical trials/latex/permutationCIwidth.jpg}
\caption{\label{fig:permCIwidth}Boxplot of the permutation-based 95\%
Confidence Interval Width estimating the Sample Average Treatment Effect
for each of 20,000 simulated observed datasets. Observed outcomes were
generated as the predicted three month Hemoglobin A1c plus a treatment
effect (if allocated to treatment) and a random residual. The horizontal
line in red is the median treatment effect Confidence Interval Width
under the pre-specified Ordinary Least Square model and Block
Randomization (See Figure 4).}
\end{figure}

\begin{figure}
\centering
\includegraphics{~/Dropbox/statistics/work/vandy/RandomizationMethods/journal-clinical trials/latex/modelAdjCIwidth.jpg}
\caption{\label{fig:adjCIwidth}Boxplot of the model-based 95\% Confidence
Interval Width estimating the Sample Average Treatment Effect, adjusted
for all baseline covariates, for each of 20,000 simulated observed
datasets. The Ordinarly Least Squares (OLS) model was pre-specified to
include all baseline covariates with restricted cubic splines on each
continuous outcome and accounted for multiple imputations.}
\end{figure}

\begin{figure}
\centering
\includegraphics{~/Dropbox/statistics/work/vandy/RandomizationMethods/journal-clinical trials/latex/trtBalance.jpg}
\caption{\label{fig:trtDiff}Boxplot of the difference in end-of-study
treatment allocation for each of 20,000 simulated allocation sequences
per allocation scheme.}
\end{figure}

\begin{table}[!h]

\caption{\label{tab:relEff}Median and 95 Percentile Confidence Interval of difference in effective sample size relative to Block Randomization for each of 20,000 generated observed datasets.  Two Block Randomization sequences per generated observed dataset to allow comparing one instance of Block Randomization to another.  Schemes are ordered by efficiency gains for permutation-based inference.}
\resizebox{\linewidth}{!}{
\begin{tabular}[t]{>{\bfseries\raggedright\arraybackslash}p{3cm}l>{\centering\arraybackslash}p{2.3cm}>{\centering\arraybackslash}p{2.3cm}>{\centering\arraybackslash}p{2.3cm}>{\centering\arraybackslash}p{2.3cm}}
\toprule
\multicolumn{2}{c}{ } & \multicolumn{2}{c}{Permutation Estimate} & \multicolumn{2}{c}{Adjusted Model Estimate} \\
\cmidrule(l{2pt}r{2pt}){3-4} \cmidrule(l{2pt}r{2pt}){5-6}
Covariate-Adjustment & Scheme & Relative Efficiency & Effective Change in N & Relative Efficiency & Effective Change in N\\
\midrule
None & Block & 1.000 & 0 (-19, 20) & 1.000 & 0 (-23, 23)\\
\cmidrule{1-6}
 & Stratified & 1.067 & 71 (6, 141) & 1.002 & 2 (-20, 25)\\
\cmidrule{2-6}
 & Urn & 1.104 & 112 (44, 183) & 1.004 & 4 (-18, 27)\\
\cmidrule{2-6}
 & Seq Matched (20p) & 1.112 & 122 (35, 225) & 1.006 & 6 (-15, 29)\\
\cmidrule{2-6}
 & Seq Matched & 1.119 & 129 (37, 242) & 1.010 & 10 (-10, 32)\\
\cmidrule{2-6}
 & Seq Rematched & 1.187 & 209 (62, 385) & 1.014 & 15 (-4, 37)\\
\cmidrule{2-6}
 & Matched (Single Batch) & 1.215 & 244 (128, 388) & 1.016 & 16 (-3, 38)\\
\cmidrule{2-6}
 & Min SE (BI, Assign) & 1.277 & 323 (109, 537) & 1.016 & 16 (-3, 39)\\
\cmidrule{2-6}
 & Min SE (Atk, 2/3 prob) & 1.326 & 388 (278, 508) & 1.018 & 19 (1, 40)\\
\cmidrule{2-6}
\multirow{-9}{3cm}{\raggedright\arraybackslash All Baseline Covariates} & Min SE (Atk, Assign) & N/A & N/A & 1.018 & 19 (4, 39)\\
\cmidrule{1-6}
 & Urn & 1.139 & 152 (79, 229) & 1.002 & 2 (-20, 25)\\
\cmidrule{2-6}
 & Seq Matched (20p) & 1.244 & 280 (156, 431) & 1.004 & 4 (-18, 27)\\
\cmidrule{2-6}
 & Min SE (BI, Assign) & 1.263 & 305 (128, 518) & 1.006 & 6 (-16, 29)\\
\cmidrule{2-6}
 & Stratified & 1.275 & 321 (189, 484) & 1.005 & 5 (-17, 28)\\
\cmidrule{2-6}
 & Seq Matched & 1.308 & 363 (222, 539) & 1.006 & 6 (-16, 28)\\
\cmidrule{2-6}
 & Seq Rematched & 1.347 & 416 (213, 663) & 1.008 & 8 (-13, 31)\\
\cmidrule{2-6}
 & Min SE (Atk, 2/3 Prob) & 1.361 & 436 (316, 572) & 1.007 & 7 (-15, 30)\\
\cmidrule{2-6}
 & Matched (Single Batch) & 1.408 & 503 (331, 711) & 1.010 & 10 (-12, 32)\\
\cmidrule{2-6}
\multirow{-9}{3cm}{\raggedright\arraybackslash Site and Most Predictive Baseline Covariates} & Min SE (Atk, Assign) & N/A & N/A & 1.001 & 1 (-13, 20)\\
\bottomrule
\end{tabular}}
\end{table}

\hypertarget{adaptive-monitoring-using-second-generation-p-values-draft}{%
\chapter{Adaptive Monitoring using Second Generation p-Values
Draft}\label{adaptive-monitoring-using-second-generation-p-values-draft}}

\hypertarget{introduction-2}{%
\section{Introduction}\label{introduction-2}}

Conclusive clinical trials either rule out clinically trivial or
clinically actionable treatment effects. Like driving blind, this is a
hard ideal to achieve without adaptive monitoring. The costs of ending
too soon, when resources and knowledge were available otherwise, can be
extraordinary (Pocock and Stone
\protect\hyperlink{ref-Pocock:2016ca}{2016}\protect\hyperlink{ref-Pocock:2016ca}{b}).
On the other hand, a significant yet non-clinically actionable study can
also be costly (Pocock and Stone
\protect\hyperlink{ref-Pocock:2016ey}{2016}\protect\hyperlink{ref-Pocock:2016ey}{a}).
For these reasons, and to the extent possible, investigators turn to
adaptive monitoring designs because the sample size is ``not too big,
not too small, but just right (Broglio, Connor, and Berry
\protect\hyperlink{ref-Broglio:2014fr}{2014}).'' To this we add the
imperative: \emph{to the point of being clinically conclusive}.

Unlike many study design aspects, clinical relevance is not an unknown
assumption. A-priori scientific relevance informs which treatment
effects are trivially null effects and which are clinically actionable
enough to change clinical practice. Most sample size estimates already
incorporate the latter into the study design. Though some study designs
rule out a set of trivially null effects (Kruschke
\protect\hyperlink{ref-Kruschke:2013jy}{2013}; Hobbs and Carlin
\protect\hyperlink{ref-Hobbs:2008ce}{2008}; Freedman, Lowe, and
Macaskill \protect\hyperlink{ref-Freedman:1984wz}{1984}), more common
practice is to test for any difference from the point null.

Trivial effects surround the point null and include, in the least,
indistinguishable treatment effects due to rounding error (Blume et al.
\protect\hyperlink{ref-Blume:SGPV}{2018}). They may also include
clinically irrelevant changes in a biomarker. This set of effects has
many names including but not limited to \emph{Indifference Zone} and
\emph{Region of Practical Equivalence} (Blume et al.
\protect\hyperlink{ref-Blume:SGPV}{2018}; Kruschke
\protect\hyperlink{ref-Kruschke:2013jy}{2013}); we call this set of
effects the \emph{Trivial Zone}. Switching from a point null to interval
null carries intuitive clinical interpretation and endows a study with
desirable statistical benefits.

Interval null hypotheses reduce family-wise Type I Error rates (Blume et
al. \protect\hyperlink{ref-Blume:SGPV}{2018}; Kruschke
\protect\hyperlink{ref-Kruschke:2013jy}{2013}). False discoveries most
frequently occur closest to the point null hypothesis, and an interval
null provides a stricter rejection criteria buffer that eliminates many
false discoveries. For the same reason, the interval null provides a
natural Type I Error adjustment for multiple looks / comparisons.
Inferential approaches to evaluating interval hypotheses include
Bayesian, Likelihood, and second-generation \emph{p}-value inference.
For any estimated interval (i.e.~Confidence Interval, Credible Interval,
Support Interval, etc.), the second-generation \emph{p}-value draws
conclusions from how much of the interval overlaps an interval
hypothesis (Blume et al. \protect\hyperlink{ref-Blume:SGPV}{2018}).

Many Bayesian adaptive trial designs incorporate interval null
hypotheses (Freedman, Lowe, and Macaskill
\protect\hyperlink{ref-Freedman:1984wz}{1984}; Berry et al.
\protect\hyperlink{ref-Berry:2010fr}{2010}; Hobbs and Carlin
\protect\hyperlink{ref-Hobbs:2008ce}{2008}; Kruschke
\protect\hyperlink{ref-Kruschke:2013jy}{2013}), and we develop an
analogous Second Generation p-value adaptive design. The rest of the
paper follows as: establishing a-priori clinically relevant guideposts,
introducing the second generation \emph{p}-value, adaptively monitoring
with with the second generation p-value, comparing the second generation
\emph{p}-value and Bayesian adaptive monitoring with the REACH clinical
trial data, and drawing final conclusions.

\hypertarget{clinically-relevant-guideposts}{%
\section{Clinically Relevant
Guideposts}\label{clinically-relevant-guideposts}}

In a two-sided study, four boundaries provide clinical guideposts.
Highly actionable treatment effects are of a magnitude of at least of
\(\delta_E\) for benefit or of \(\delta_H\) for harm. Trivial treatment
effects are between \(\delta_{TE}\) and \(\delta_{TH}\) and encompass
the point null (Figure 1). The remaining effects are moderately
actionable; moderate treatment benefits are between \(\delta_{TE}\) and
\(\delta_{E}\) and moderate treatment harms are between \(\delta_{TH}\)
and \(\delta_{H}\). In any study, the implementation of an intervention
takes into account secondary measures such as side effects, costs, etc..
Hence, a highly actionable effect is likely to outweigh the off-setting
benefits / costs. A moderately actionable effect has greater equipoise
with offsetting benefits / costs.A one sided study omits \(\delta_H\)
and \(\delta_{TH}\) to focus only on \(\delta_TE\) and \(\delta_{E}\).

\begin{figure}[H]

{\centering \includegraphics{~/Dropbox/statistics/work/vandy/AdaptiveMonitoring/manuscript/figs/clinicalGuideposts/Slidev03_01} 

}

\caption{Clinically relevant guideposts are determined during study design, based on scientific context, and ought to be incorporated into the final study inference.  In a one-sided study (left figure), the clinical guideposts create the regions: no more than trivial effect, moderately actionable effect, and highly actionable effect.  In a two-sided study (right figure), three regions are created: trivial effects, moderately actionable effects, and highly actionable effects.}\label{fig:clinicalGuideposts}
\end{figure}

A similar set of clinical guideposts, the \emph{Region of Equivalence},
uses only two boundaries(Freedman, Lowe, and Macaskill
\protect\hyperlink{ref-Freedman:1984wz}{1984}; Hobbs and Carlin
\protect\hyperlink{ref-Hobbs:2008ce}{2008}). Treatment effects outside
the \emph{Region of Equivalence} deem the novel intervention as either
clinically superior or inferior to the standard of care. Under certain
conditions (a one-sided study with a buffer around the point null),
these guideposts match the one-sided guideposts we propose. The
\emph{Trivial Zone} is necessary to rule out effects trivial to the
point null and receive the benefits of an interval null such as reducing
the False Discovery Probability.

Setting a-priori clinical guideposts brings transparency to the study
design. Most study designs already incorporate \(\delta_E\) when
determining an adequate study sample size. Yet, in many instances, end
trial inference does not incorporate \(\delta_E\) (or other clinical
guidepost decisions). For example, without knowing \(\delta_E\), none of
a Confidence, Credible, or Support Interval inform original study design
intentions.

Excellent references are available for helping establish clinical
guideposts (Freedman, Lowe, and Macaskill
\protect\hyperlink{ref-Freedman:1984wz}{1984}; Spiegelhalter, Freedman,
and Parmar \protect\hyperlink{ref-Spiegelhalter:1994cn}{1994}; Kruschke
\protect\hyperlink{ref-Kruschke:2018bz}{2018}; Blume et al.
\protect\hyperlink{ref-Blume:SGPV}{2018}).

\hypertarget{second-generation-p-value}{%
\section{Second Generation p-value}\label{second-generation-p-value}}

An inferential metric, the second-generation \emph{p}-value indicates
when the data are compatible with the alternative hypothesis, the
\emph{Trivial Zone} null hypothesis, or when the data are inconclusive
(Blume et al. \protect\hyperlink{ref-Blume:SGPV}{2018}). More generally,
it may be used for any interval hypothesis. The second-generation
\emph{p}-value calculates the overlap between an interval \emph{I} (any
interval including but not limited to a Confidence, Credible, Support
Interval, etc.) and the set of effects \(\Delta_H\) in the hypothesis
\emph{H}. The interval includes {[}\emph{a}, \emph{b}{]} where \emph{a}
and \emph{b} are real numbers such that \emph{a} \textless{} \emph{b},
and the length of the interval is \emph{b} - \emph{a} and denoted
\textbar{}\emph{I}\textbar{}. The overlap between the interval and the
set \(\Delta_H\) is \(|I \bigcap \Delta_H |\). The second generation
p-value is then calculated as

\[
p_{H} = \frac{| I \bigcap \Delta_H |}{|I|} \times max\left\{\frac{| I |}{2 | \Delta_H |}, 1 \right\}.
\] The multiplicative factor,
\(max\left\{\frac{| I |}{2 | \Delta_H |}, 1 \right\}\), provides a small
sample size correction -- setting \(p_H\) to 0.5 when an interval
overwhelms \(\Delta_H\) by at least twice the length. For a
\emph{Trivial Zone} null hypothesis \emph{T}, trivially null effects are
ruled out when the \(p_T = 0\) whereas non-trivial effects are ruled out
when \(p_T = 1\). The data are inconclusive when \(0 < p_T < 1\).

Based on the four clinical guideposts, we define and focus on a Highly
Actionable Hypothesis, \emph{HA}, and Trivial Hypothesis, \emph{T} .

\begin{itemize}
\item
  Hypothesis \emph{HA}: The treatment effect lies within a \emph{Region
  of Clinically Highly Actionable Effects}
  \(\Delta_{HA} = (-\infty, \delta_{H}] \cup [\delta_{E}, \infty)\) for
  a two-sided study and \([\delta_{E}, \infty)\) for a one-sided study
  where a positive benefit is beneficial. Where a negative effect is
  beneficial, the one-sided study sets
  \(\Delta_{HA} = (-\infty, \delta_{E}]\).
\item
  Hypothesis \emph{T}: The treatment effect lies within a \emph{Region
  of Clinically Trivial Effects}
  \(\Delta_T = [\delta_{TH}, \delta_{TE}]\) for a two-sided study. In a
  one-sided study where a positive effect is beneficial,
  \(\Delta_T = (\infty, \delta_{TE}]\) and is called a \emph{Region of
  At Most Clinically Trivial Effects}. The bounds are mirrored when a
  negative effect is benificial.
\end{itemize}

Neither of these two hypotheses include moderately actionable treatment
effects. When applied to the four clinical guideposts, nine conclusions
may be drawn from the second-generation \emph{p}-value and the Regions
of Clinically Highly Actionable Effects and Clinically Trivial Effects
(Figure 2). To motivate our adaptive monitoring design, we reduce to
three conclusions:

\begin{itemize}
\tightlist
\item
  When \(p_{HA} = 0\), the treatment effect is not clinically highly
  actionable.
\item
  When \(p_{T} = 0\), the treatment effect is not trivial different from
  the point null.
\item
  Otherwise, the treatment effect is inconclusive
\end{itemize}

Again, the interpretation of these regions closely relate to the
interpretations of the \emph{Region of Equivalence}. They match exactly
when in the case of a one-sided study with a \emph{Trivial Zone} around
the point null.

\begin{figure}[H]

{\centering \includegraphics{~/Dropbox/statistics/work/vandy/AdaptiveMonitoring/manuscript/figs/clinicalGuideposts/Slidev03_02} 

}

\caption{Final study inference ought to incorporate a-priori clinical guideposts.  The second generation p-value draws inference based on hypothesized sets of treatment effects.  We focus on two sets of hypotheses that form the Region of Clinically Trivial Effects and the Region of Clinically Highly Actionable Effects. (This figure focuses on two-sided studies yet similar conclusions are drawn for one-sided studies). With the two sets of hypotheses, the second-generation p-value can rule out trivial effects (the top four conclusions), rule out non-highly actionable effects (the middle five conclusions), or declare the study yet inconclusive.  Confidence Intervals that correspond to a p-value close to but not exceeding 0.05 would be declared inconclusive.}\label{fig:figs}
\end{figure}

\hypertarget{adaptive-monitoring-rules-guidance}{%
\section{Adaptive Monitoring Rules /
Guidance}\label{adaptive-monitoring-rules-guidance}}

A study implements the following rules for adaptive monitoring with the
second-generation \emph{p}-value:

\begin{itemize}
\item
  Design: Investigators a-priori determine the four clinical guideposts
  (two guideposts in the case of a one-sided study).
\item
  Wait to Monitor: Enroll \emph{B} patients before applying monitoring.
\item
  Monitor: Calculate the second-generation p-value using an inferential
  interval of choice. Raise an alert when \(p_{HA} = 0\) or \(p_T = 0\).
  Continue monitoring until affirming the \emph{same} alert (i.e.~that
  again \(p_{HA} = 0\) or \(p_T = 0\)) \emph{K} patients later.
\item
  Stop: Stop once affirming an alert or at the end of resources.
\item
  Report: Report only the interval at stopping.
\end{itemize}

Finding the appropriate \emph{B} and \emph{K} is done through
simulations in the study design stage; both help protect against Type I
Errors and bias. Chapter 3 provides practical guidance for determining
\emph{B} and \emph{K}. Type I Errors more commonly occur early in
adaptive monitoring while estimated statistics are yet unstable, and
they may occur randomly in the discrete Brownian motion of the
Monitoring Interval. Bias is inherent in all adaptive monitoring schemes
that stop at the first instance an alert is raised. Requiring \emph{K}
patients to affirm an alert allows regression to the true treatment
effect and improves the reported interval's coverage rate. To draw
emphasis, we call intervals used in the monitoring phase Monitoring
Intervals as they are not to be interpreted.

When the true effect is moderately actionable, the study may stop for
concluding the effect to be either not-trivial or not-highly actionable.
This may bring consertnation; however, it is an important study design
feature. Without a region of clinically moderately actionable effects, a
highly actionable effect would border trivial effects and have a 50-50
chance of stopping for being actionable or trivial. A study that stops
for concluding a non-trivial effect may include moderately actionable
effects in the inferential interval. And similarly, moderately
actionable effects may be included in inferential intervals when
stopping to conclude a non-actionable effect. This behavior reflects the
greater degree of equipoise between the moderately actionable effects
and the off-setting harms/benefits.

Moderately actionable effects have a greater chance of raising
conflicting alerts -- for example to raise an alert for a non-trivial
effect then \emph{K} patients later alert for a non-highly actionable
effect. For this reason, we require the affirmation alert to be the same
as the alert it affirms. Only the final interval's operating
characteristics are relevant.

With only adapting the monitoring rules, the remaining rules apply to
monitoring with Bayesian Credible Intervals. An alert for a non-highly
actionable effect when P(Treatment Effect \(\notin \Delta_{HA}\)
\textbar{} Data) \textgreater{} 1 --
\(\alpha_{criteria-not-actionable}\) or for a non-trivial effect when
P(Treatment Effect \(\notin \Delta_{T}\) \textbar{} Data) \textgreater{}
1 -- \(\alpha_{criteria-not-trivial}\).
\(\alpha_{criteria-not-actionable}\) and
\(\alpha_{criteria-not-trivial}\) are study design tuning parameters
based on simulations to achieve a desired end of study Type I Error and
Power.

\hypertarget{adaptively-monitoring-the-reach-clinical-trial}{%
\section{Adaptively Monitoring the REACH clinical
trial}\label{adaptively-monitoring-the-reach-clinical-trial}}

\hypertarget{context}{%
\subsection{Context}\label{context}}

The Rapid Education/Encouragement And Communications for Health (REACH)
randomized clinical trial (Nelson et al.
\protect\hyperlink{ref-Nelson:2018bw}{2018}) is designed to help
patients with diabetes better manage glycemic control (as measured by
Percent Hemoglobin A1c) and adhere to medication. Patients randomized to
the intervention receive text message-delivered diabetes support for
over 12 months. Now with complete enrollment, 512 patients are currently
being followed up through 3, 6, 12, and 15 months.

In this population, lower Hemoglobin A1c reflects improved glycemic
control. A change in Hemoglobin A1c of +/- of 0.15 (median REACH
baseline Hemoglobin A1c of 8.20 {[}IQR of 7.20, 9.53{]}) is clinically
trivial, whereas a decrease of Hemoglobin A1c of 0.5 is highly
actionable to the point of adopting this novel intervention. While we
anticipate the intervention to improve Hemoglobin A1c, we designed the
study to be two-sided. That is, \(\delta_E\) = -0.50, \(\delta_{TE}\) =
-0.15, \(\delta_{TH}\) = 0.15, and \(\delta_H\) = 0.50.

\hypertarget{simulation}{%
\subsection{Simulation}\label{simulation}}

The performance of adaptive monitoring using the second generation
p-value and posterior probabilities was observed from 20,000 bootstrap
samples REACH patients. With block two randomization half of the
patients were randomized to receive the REACH intervention. For
outcomes, we used three month predicted Hemoglobin A1c from an ordinary
least square model adjusting for baseline covariates including
Hemoglobin A1c, age at enrollment, gender, years of education, years of
diabetes, race / ethnicity, medication type, income, and type of
insurance. All continuous baseline covariates flexibly allowed for
non-linear associations through restricted cubic splines. Predicted
Hemoglobin A1c changed by a treatment effect for those randomized to the
REACH intervention. We simulated multiple treatment effect settings
\{Treatment Effect: -1, -0.75, -0.50, -0.375, -0.15, 0\}. Only negative
effects were investigated but by symmetry, increases in Hemoglobin A1c
of the same magnitude perform similarly. In these simulations we assumed
instantaneous outcomes.

Monitoring began after the 40th enrolled patient and continued every
20th patient until either affirming an alert 40 patients later or until
reaching the end of resources (512 patients). For each second-generation
\emph{p}-value Monitoring Interval, we fit a marginal ordinary least
squares regression of outcome Hemoglobin A1c given treatment assignment
and obtained the 95\% Confidence Interval for the treatment effect. And,
for adaptive monitoring using posterior probabilities, we used a
Bayesian alternative to the t-test: two-sample difference of means where
the intervention group mean, \(Y_I\), and control group mean, \(Y_C\),
were both distributed \(t(\mu, \sigma, \nu)\) (Kruschke
\protect\hyperlink{ref-Kruschke:2013jy}{2013}).

For Bayesian priors, we set \(\mu\) \textasciitilde{} \(N(\bar{y}\), 1 /
\(\phi\)-1(.9)) as a fairly flat prior centered on the average mean
Hemoglobin A1c of all observations with a 0.10 probability of observing
an absolute change in Hemoglobin A1c greater than 1. For a skeptical
prior on \(\mu\), we mixed the flat prior 1:1 with \(\mu\)
\textasciitilde{} \(N(\bar{y}\), 0.15 / \(\phi\)-1(0.95)). We set
\(\sigma\) \textasciitilde{} Gamma(\(s_y\) / 1000, \(s_y\) 1000) where
\(s_y\) was the non-pooled standard deviation of all Hemoglobin A1c
outcomes, and \(\nu\) \textasciitilde{} \(Gamma(1 / 600, 30) + 1\). The
\(\nu\) parameter places a prior on the skewness of the data, and was
chosen such that with the flat prior yielded group means similar to raw
means.

The two Bayesian adaptive designs (one with a flat and the other
skeptical prior) were calibrated to the second generation \emph{p}-value
design by finding the
\(\alpha_{criteria-not-actionable} = \alpha_{criteria-not-trivial}\)
that achieve the same Type I Error given the true effect is zero and
that achieve the same probability of stopping for a not-trivial effect
given the true effect is -0.15 (the border of trivial effects). The
later calibration was then used to compare adaptive designs in terms of
reasons for stopping, average trial sample size, bias, and coverage of
the reported interval. Operating characteristics are estimated for given
treatment effects. When allowing for distributional assumptions of
observing the treatment effect, a correctly specified prior has zero
bias.

\hypertarget{results}{%
\subsection{Results}\label{results}}

The flat and skeptical prior Credible Interval adaptive monitoring
designs were calibrated to have the same Type I Error as adaptive
monitoring with the second-generation \emph{p}-value (Type I Error =
0.049; \textbf{Figure 3.7}). While this was close to 0.05, the Type I
Error changes depending on the wait time until monitoring and the number
of looks. Adaptive monitoring using the second-generation \emph{p}-value
and flat prior Credible Interval were near identical in Power.
\textbf{whereas, adaptive monitoring using a skeptical prior Credible
Interval achieved less Power. The same trend was seen in all other
results.}

The probability of stopping for being not-trivial provides an adaptive
monitoring analog for a Type I error under an interval null hypothesis.
When the true treatment effect was truly -0.15 (i.e.~the boundary of
trivial effects), and when adaptively monitoring with second-generation
\emph{p}-values, the probability of stopping for concluding not-trivial
was \emph{0.015} (\emph{Figure 4}). Posterior probability monitoring
designs were again calibrated to this same probability for this and the
remaining figure results. The probability of stopping for being
not-trivial was slightly higher when monitoring on the flat prior
Credible Interval. However, for a clinically meaningful treatment effect
of -0.50, monitoring on the flat prior Credible Interval had a higher
probability of stopping for concluding a non-highly-actionable effect
than monitoring with the second generation \emph{p}-value. The two
designs that monitored on Credible Intervals were more likely to stop
for concluding not-highly actionable effects than monitoring on the
second-generation \emph{p}-value; this is consistent with both priors
being at least slightly favorable to the point null. \emph{This was
dramatically true for monitoring on the skeptical prior Credible
Interval. However, the skeptical prior Credible Interval design could be
re-tuned (such as having a longer burn in) to improve performance.}

Trials lasted longer when the treatment effect was moderately-actionable
(\emph{Figure 5}). \emph{The exception was when monitoring with the
skeptical prior Credible Interval. Because treatment effects were drawn
to the null early on, treatment effects were estimated to be outside the
\emph{Clinically Highly Actionable Region} before the data could
adequately overwhelm the prior. As a consequence, the bias and coverage
suffered greatly when monitoring with the skeptical prior Credible
Interval (Figures 6 and 7).}

The bias was well mitigated when adaptively monitoring with the second
generation \emph{p}-value and a posterior conditioned on a flat prior
(Figure 6). The flat prior Credible Interval pulls estimates toward the
point null, which tended to slightly help when the treatment effect was
clinically meaningful and slightly hurt when the trivially null. Neither
the flat nor skeptical prior are biased if they are each reflective of
the true distribution of observing treatment effects. Coverage rates
were slightly higher when monitoring with the flat prior Credible
Interval compared to the second generation \emph{p}-value. Of the
treatment effects investigated, the worst coverage was \emph{0.93} when
monitoring with the second generation \emph{p}-value.

\begin{figure}[H]

{\centering \includegraphics[width=3.5in]{~/Dropbox/statistics/work/vandy/AdaptiveMonitoring/manuscript/figs/powerCalibratedLooks20} 

}

\caption{In the REACH target population, a decrease in Hemoglobin A1c reflects desireable improvement in glycemic control.  The Power curve (i.e. the probability of rejecting the point null of 0), was estimated from 20,000 adaptive monitoring simulations when monitoring using the second generation p-value (red circle) and posteriors conditioning on a flat prior (blue triangle) and skeptical prior (purple square).  The intervention was simulated to have an effect of -1 (highly beneficial), -0.75, -0.50, -0.375, -0.15, to 0 effect (the point null).  Bayesian adaptive monitoring designs were calibrated to have the same Type I Error as the second generation p-value adaptive design.}\label{fig:Power}
\end{figure}

\begin{figure}[H]

{\centering \includegraphics[width=2in]{~/Dropbox/statistics/work/vandy/AdaptiveMonitoring/manuscript/figs/probEffLooks20} \includegraphics[width=2in]{~/Dropbox/statistics/work/vandy/AdaptiveMonitoring/manuscript/figs/probFutLooks20} \includegraphics[width=2in]{~/Dropbox/statistics/work/vandy/AdaptiveMonitoring/manuscript/figs/probMoreLooks20} 

}

\caption{In adaptive monitoring, a study can end in one of three states: concluding non-trivial effect, concluding non-highly-actionable effect, or ending at the end of resources.  Above are estimated probabilities of ending in each of these states for each design and treatment effect.  The probability of concluding non-trivial is an interval null analog to Power.  The Bayesian adaptive monitoring designs were calibrated to have the same probability of concluding a non-trivial effect as adaptive monitoring with the second generation p-value.  All following results are based off this calibration.}\label{fig:state}
\end{figure}

\begin{figure}[H]

{\centering \includegraphics[width=3.5in]{~/Dropbox/statistics/work/vandy/AdaptiveMonitoring/manuscript/figs/averageNLooks20} 

}

\caption{Across  simulations, the average time to stopping for the three adaptive designs is shown above.  In these simulations the earliest possible stopping time was the 80th patient.  Monitoring began at the 40th patient and continued at every 20th patient.  Stopping required raising an alert and affirming the alert 40 patients later.  Moderately actionable treatment effects required the greater sample size, while highly actionable and trivial effects required smaller sample sizes to suggest stopping.}\label{fig:sampleSize}
\end{figure}

\begin{figure}[H]

{\centering \includegraphics[width=3.5in]{~/Dropbox/statistics/work/vandy/AdaptiveMonitoring/manuscript/figs/biasLooks20} 

}

\caption{At the study end, the final estimate and interval are reported.  Across simulations, the bias was well mitagated when adaptively monitoring using the second generation p-value and posterior probability conditioned on a flat prior.  A positive bias occurs when being pulled toward the null, as happens with the skeptical prior.}\label{fig:bias}
\end{figure}

\begin{figure}[H]

{\centering \includegraphics[width=3.5in]{~/Dropbox/statistics/work/vandy/AdaptiveMonitoring/manuscript/figs/coverage} 

}

\caption{At the study end, the final estimate and interval are reported.  In these simulations, a 95\% Confidence Interval was reported when adaptively monitoring with the second-generation p-value (though any other interval could have been used for monitoring and reporting).  And, Credible Intervals are reported for the Bayesian adaptive designs.}\label{fig:coverage}
\end{figure}

\hypertarget{discussion}{%
\section{Discussion}\label{discussion}}

We developed an adaptive monitoring scheme, using the second generation
\emph{p}-value, to follow studies until either ruling out trivially null
or clinically highly actionable treatment effects. Two major
contributions come from this scheme: (1) the monitoring scheme
incorporates clinically relevant guideposts that may help trivially null
and clinically highly actionable studies stop quickly while still
reaching a clear clinical conclusion and (2) the easy-to-calculate
second generation \emph{p}-value allows an investigator to adaptively
monitor with their inferential interval of choice (including but not
limited to Confidence Intervals, Credible Intervals, and Support
Intervals). While not explored explicity in this paper, this design is
well suited for following multiple outcomes and following subgroups
until they reach clear clinical conclusions.

As seen in other similar designs, the use of an interval null decreases
the False Discovery Rate, making the results more reproducible (Blume et
al. \protect\hyperlink{ref-Blume:SGPV}{2018}; Kruschke
\protect\hyperlink{ref-Kruschke:2013jy}{2013}). Incorporating clinical
relevance into the inference brings study design transparency that
otherwise may get lost. For example, a Confidence, Credible, or Support
Interval alone do not inform of the targeted treatment effect when
choosing a sample size.

With data from the REACH randomized clinical trial, we simulated
adaptive monitoring using rules based on the second-generation
\emph{p}-value compared to rules based on the posterior probabilities.
Monitoring with the second-generation \emph{p}-value and posterior
probabilities conditioned on a flat prior performed similarly in terms
of Power, probability of stopping for efficacy, average sample size,
bias, and reported interval coverage. Monitoring using a flat prior
Credible Interval had a slight edge in these metrics but also a slightly
worse probability of concluding highly-actionable effects were
not-highly actionable. The flat prior was not uniformly flat, it
slightly favored the point null.

\emph{In these simulations, adaptive monitoring with a skeptical prior
Credible Interval did not perform well. The skeptical prior pulled
estimates close to the null such that simulated adaptive trials stopped
before being overwhelmed by the data. However, a longer burn or an
alternative tuning to \(\alpha_{criteria-not-actionable}\) would
ameliorate the skeptical prior Credible Interval design.}

All of the adaptive designs studied may be modified to achieve a desired
Type I Error and Power. When adaptively monitoring with the second
generation \emph{p}-value, one may change the wait time before
monitoring and number of looks. We encourage clinical guideposts to be
anchored on clinical relevance and not change solely for the purpose of
achieving statistical properties. The same encouragement to not change
anchors denoting clinical relevance hold true for monitoring with
posterior probabilities.

\hypertarget{the-sgpvam-package-and-practical-recommendations-for-adaptive-monitoring-with-the-second-generation-p-value}{%
\chapter{The sgpvAM Package and Practical Recommendations for Adaptive
Monitoring with the Second Generation
p-value}\label{the-sgpvam-package-and-practical-recommendations-for-adaptive-monitoring-with-the-second-generation-p-value}}

\hypertarget{introduction-3}{%
\section{Introduction}\label{introduction-3}}

In Paper 2 (Chapter 3) we present an adaptive monitoring scheme that
follows studies until evidence supports either a non-trivial or
non-highly actionable treatment effect. The design is very easy to
implement. It can done by anyone who can think about the clinical
interpretation of possible effect sizes and calculate an interval
estimate for their effect, such as a support or confidence interval
(CI). However, estimating the operating characteristics of a given study
design is not as easy. Without tools to assist them, it could be a
barrier to implementation of the method. In practice, the trialist will
want to know the operating characteristics under the following adaptive
monitoring design features.

\begin{itemize}
\tightlist
\item
  Frequency of looks at the data, i.e.~recalculate CI at every jth
  subject.
\item
  Minimum precision requirement before applying monitoring rules,
  i.e.~check monitoring rules only if \textbar{}CI\textbar{} \textless{}
  w.
\item
  Required observations between an alert and affirmation to stop,
  i.e.~evaluate stopping rule j*k subjects after alert.
\item
  Anticipated maximum amount of data that could be collected, maxN,
  i.e.~the sample size at which the study will cease collection
  regardless of the stopping rules.
\item
  Lag time between enrolling a subject and observing their outcome
  measured in the number of subjects recruited during the lag, i.e.~m
  additional subjects will be recruited in the time between one subject
  being recruited and that one subject's outcome being observed. The
  traditional trialist will mainly be looking for the point null Type I
  error probability and Power, i.e.~the probability of concluding an
  effect is non-trivial for a given true effect size. They may also want
  some simple summary statistics for the potential sample sizes. We hope
  to provide access to these and to a more rich set of operating
  characteristics. Operating characteristics we can potentially estimate
  for a given set of design features include the following.
\item
  Distribution of potential sample sizes.
\item
  Point Null Type I error, when the point null is true, i.e.~probability
  of excluding the point null from the final interval estimate.
  Classical trialists will want reassurance this is \textless{} 5\%.
\item
  Interval Null Type I error, when the point null is true,
  i.e.~probability of excluding the entire trivial zone from the final
  interval estimate. This will be less than or equal to the point null
  Type I error.
\item
  Power vs the point null, i.e.~probability the final interval estimate
  excludes the point null for a given true effect size. This is akin to
  classical statistical power.
\item
  Power vs the interval null, i.e.~probability the final interval
  estimate excludes the null zone for a given true effect size. This
  will be less than or equal to the point null power, but is
  conceptually the preferable quantity. In better terms, this is the
  probability of concluding the effect is non-trivial for given true
  effect sizes.
\item
  Probability of concluding effect is non-highly actionable for given
  true effect sizes.
\item
  Probability of an inconclusive finding at the end of resources. Note a
  clinically inconclusive finding is a possibility whenever maxN is
  finite and/or m \textgreater{} 0.
\item
  Bias, MSE, and interval coverage probability from a frequentist
  perspective.
\item
  False confirmation probability under a specified prior distribution.
  To address these needs, we provide an R function, sgpvAM, that
  simulates the above design operating characteristics for normal and
  binomial outcomes, and we offer practical advice setting the minimum
  wait time (in terms of inferential interval width) and the number of
  looks before affirming an alert.
\end{itemize}

\hypertarget{sgpvam-package}{%
\section{sgpvAM Package}\label{sgpvam-package}}

The sgpvAM package allows the user to obtain study design operating
characteristics under a variety of settings for adaptive monitoring
using the second generation p-value.

\hypertarget{mcmc-replicates}{%
\subsection{MCMC Replicates}\label{mcmc-replicates}}

The user may use the sgpvAM function to generate MCMC replicates of
outcomes and intervention assignments along with an estimate of the
effect and a lower- and upper- interval bound; replicates are generated
using parallel computing. Alternatively, the user may provide their own
generated data together with an estimated effect and interval bounds.
When using the sgpvAM function, the user specifies the data generation
function (any of the r{[}dist{]} such as rnorm) along with arguments to
the function. Similarly, the user specifies effect generation. At this
point, only fixed effects have been thoroughly tested. However, by
specifing a distribution for the effects, the user may explore False
Discovery Probabilities and other operating characteristics, such as
bias, dependent on distributional assumptions of the effect.

\hypertarget{one--vs-two-sided-hypotheses}{%
\subsection{One- vs Two-Sided
Hypotheses}\label{one--vs-two-sided-hypotheses}}

Clinical Guideposts defining regions of Trivial and Highly Actionable
Effects must be provided though may be one- or two-sided. The point null
must be within the Trivial Region and cannot be a boundary of the
region. For general nomenclature, inputs to define the regions are:
deltaL2 (the Clinically Highly Actionable Boundary less than the point
null), deltaL1 (the Trivial Region Boundary less than the point null),
deltaG1 (the Trivial Region Boundary greater than the point null), and
deltaG2 (the Clinically Highly Actionable Boundary greater than the
point null). See Paper 2 (Chapter 3) for a thorough discussion of these
regions.

\hypertarget{tuning-study-parameters}{%
\subsection{Tuning study parameters}\label{tuning-study-parameters}}

To maximize performance of operating characteristics under a given
sample size the sgpvAM function allows the user to specify multiple wait
time settings, frequency of looks, and number of steps before affirming
a stopping rule. (The wait time is the time until the expected
Confidence Interval Width achieves a certain length or less).

\hypertarget{operating-characteristics-under-normal-outcomes}{%
\subsection{Operating characteristics under normal
outcomes}\label{operating-characteristics-under-normal-outcomes}}

After generating the operating characteristics under a fixed normal
outcome, the user may use the locationShift function to obtain operating
characteristics under a range of fixed treatment effects. The function
uses the saved MCMC replicates and adds to them if needed for additional
monitoring.

\hypertarget{ecdf-of-sample-size-and-bias}{%
\subsection{ECDF of sample size and
bias}\label{ecdf-of-sample-size-and-bias}}

Once a study design has been selected based on average performance
(sample size, bias, and error probabilities), the user may use the
ecdf.sgpv function to see the empirical cumulative distribution across
the MCMC replicates for sample size and bias under a specific design.
The user may see the estimated probability of the sample size exceeding
a certain maximum sample size.

\hypertarget{general-suggestions}{%
\subsection{General suggestions}\label{general-suggestions}}

Computations may be time consuming. It is recommended to start with 1000
replicates to get a general sense of average sample size and error
probabilities under a variety of investigated wait times and affirmation
steps. Investigating many wait times increases the computational burden.
It is also recommended to generate MCMC replicates in the (or one of
the) mid point(s) between the Clinically Trivial and Highly Actionable
Regions. This is the region with greatest expected sample size and
reduces the burden of the locationShift function to generate more data.

\hypertarget{inputs}{%
\subsection{Inputs}\label{inputs}}

\hypertarget{return-values}{%
\subsection{Return values}\label{return-values}}

The sgpvAM function returns a list with three elements:

\begin{enumerate}
\def\labelenumi{\arabic{enumi}.}
\tightlist
\item
  mcmcMonitoring -- the mcmcReplicates when outData is TRUE,
\item
  mcmcEndOfStudy -- operating characteristics on average and ECDF for
  each combination of wait time and number of steps before affirming a
  stopping rule
\item
  inputs -- Inputs in to the sgpv function
\end{enumerate}

As supporting material for the package, we have developed an extensive
vignette that illustrates using sgpvAM to estimate and explore the
impact of study design choices on Point Null Type I error, Power for a
range of true effect sizes, average sample sizes, the distribution of
possible sample sizes, and more. These include the types of figures and
calculations that will be of particular interest to the traditional
trialist. The full vignette may be found at
\url{https://github.com/chipmanj/sgpvAM} or by loading the sgpvAM
package and calling Vignette(package = ``sgpvAM'', topic = ``README'').
Three of the example figures are presented briefly below. Figure 1 is a
classical power curve which shows the probability the final CI will
exclude the point null at various true effect sizes. The power when the
true effect size is equal to the point null is the classical point null
Type I error probability. The figure illustrates the impact of various
wait times on the power curve. The second generation p-value adaptive
design allows for designing trials with finite and infinite sample sizes
in mind. We recommend presenting both. Figure 1 illustrates the infinite
sample size. Notice the point null Type I error is bounded below 5\%
even when the maximum sample size is theoretically infinite. Under this
framework, a study that stopped for reaching a maximum sample size could
be restarted without concern of controlling point null Type I error.
Figure 2 displays the impact of increasing the affirmation steps on the
average sample size. Increasing the affirmation steps has benefits in
reducing bias, increasing interval coverage probabilities, and
increasing the stability of conclusions particularly in the presence of
a lag between recruitment and outcome observation. These benefits come
with an increase in average sample size, particularly in between the
bounds of the Trivial and Highly Actionable regions, i.e.~where
erroneous conclusions due to stopping too early are most likely.
Non-adaptive studies are typically designed a with high probability of
an inconclusive finding. Consider a non-adaptive study designed to have
80\% power at a clinically highly actionable effect size. Such a study
will include the point null in its final interval 20\% of the time. It
will include values from the null region with an even higher
probability. Although designed to only stop when a clinically highly
actionable conclusion has been found, second generation p-value adaptive
monitored trials may yield a clinically inconclusive finding if the
trial has a fixed maximum sample size and/or a lag between recruitment
and outcome observation. This probability is highest in between the
bounds of the Trivial and Highly Actionable regions. Figure 3
illustrates the control of this probability provided through increasing
the affirmation steps in the presence of a 50 subject lag time. Even
with this large lag, a relatively small affirmation step requirement
bounds the probability of an inconclusive finding below 20\%.

\hypertarget{practical-recommendations}{%
\section{Practical Recommendations}\label{practical-recommendations}}

Of key importance to classical trialists is controlling point null Type
I Error and achieving a high probability of excluding the point null
when the true effect size is clinically highly actionable. Of key
importance to medical researchers is completing the study with a
clinically highly actionable finding. We show that focusing on the later
will achieve the former. When a study is not able to observe outcomes
immediately, care should be taken to reduce the risk of stopping and
then being inconclusive after observing the remaining observations. To
control error rates, one may change the clinical guideposts, reduce the
number of times monitoring the study, and/or increase the number of
steps before affirming a stopping rule. Ideally, the clinical guideposts
chosen for their clinical interpretation. We discourage altering them
for the sake of operational characteristics. Instead, we encourage
optimizing operating characteristics through the waiting a period of
time before monitoring and requiring a number of observations to pass
until affirming an alert to stop the study. We consider various wait
times based upon the expected confidence interval width under an assumed
outcome standard deviation. Twenty thousand MCMC replicates of a study
with standard normal outcomes are generated using the sgpvAM package
under five one-sided hypotheses settings and five symmetric two-sided
hypotheses. The settings reflect situations where (Setting 1) the
Trivial and Highly Actionable Effect bounds are both close to the point
null of zero, (Settings 2-4) the Trivial Effect bound is close to the
null and the Highly Actionable Effect is far from the point null, and
(Setting 5) situations where the bound for Highly Actionable Effects is
far from the null and the bound for Trivial Effects is increasingly
close to the Highly Actionable Effect bound. These results generalize to
normal outcomes with clinical guideposts relative to the standard
deviation. For both one- and symmetric two-sided hypotheses, the
probability of a Type I Error is minimized by waiting until the
Confidence Interval Width is the midpoint between the Trivial and Highly
Actionable Zones (Figure 4). For example, a one-sided study with At Most
Trivial Effects defined as (-\(\infty\), 0.1{]} and Highly Actionable
Effects as {[}0.2, \(\infty\)) reduces the probability of a Type I Error
by waiting until the Confidence Interval Width is 0.15. In these
simulations, the probability of a Type I Error remained less than or
equal to 0.05 when waiting longer, until the Confidence Interval Width
equaled the positive boundary of the Trivial Effects. The following
operating characteristics benefit from a longer wait time (i.e.~waiting
until the CI Width is more narrow): power (Figure 5); an interval null
equivalent to Type I Error (Figures 7 and 8), Power (Figure 9), and Type
Two Error (Figure 10); and the probability of stopping for conclusive
observed outcomes yet becoming inconclusive after observing the
remaining unobserved outcomes (Figure 11). On the other hand, a shorter
wait time yields smaller average sample sizes (Figure 6). The gains in
sample size diminishes once the wait time is the midpoint between the
Trivial and Highly Actionable Zones.\\
On whole, this suggests waiting no longer than this midpoint and may
motivate waiting slightly longer (for a smaller Confidence Interval
Width). When the boundary of the Trivial Zone is close to half the
boundary of the Highly Actionable Zone, it may be worthwhile to set the
Confidence Interval wait time to the boundary of the Trivial Zone (refer
to settings 3 and 4 in Figure 6 compared to setting 2). Across these
simulations, setting the wait width to one quarter between the Trivial
and Highly Actionable boundaries balances well maximizing operating
characteristics without a substantive increase in average observed
sample size. In the earlier one-sided example, this would be to wait
until the expected Confidence Interval Width equals 0.125. This leads us
to some practical recommendations regarding wait time. First, if the
outcome's variability is well known, set the wait time to one quarter of
the distance between the Trivial and Highly Actionable boundaries.
Second, if the variability is not well known, error on over-assuming the
variability. This will result in setting a longer-wait time than needed.
In this instance, we suggest setting the Confidence Interval Width to
the mid point between the Trivial and Highly Actionable boundaries.
Studies may have intermediate outcomes. For effects that are neither
Trivial nor Highly Actionable, there's a very plausible chance the study
ends conclusively for the outcomes observed yet becomes inconclusive
after observing the outcomes of remaining study observations (Figure
11). To reduce this risk, we recommend increasing the number of steps
required to affirm stopping rules.

\hypertarget{conclusion}{%
\section{Conclusion}\label{conclusion}}

We provide the sgpvAM package to provide greater ease of access to
develop and study the operating characteristics of adaptive monitoring
with the second generation p-value. Based on simulations of normally
distributed data, we recommend waiting until the expected Confidence
Interval Width is one quarter the absolute distance between the Trivial
and Highly Actionable Region boundaries. When outcomes do not occur
immediately, there is a risk of the study stopping then being
inconclusive once all observations are observed. This risk is reduced by
requiring an increased number of steps before affirming an alert to stop
the study.

\hypertarget{conclusions-1}{%
\chapter{Conclusions}\label{conclusions-1}}

This body of work introduces study design methods and to improve the
effective sample size, allow for greater personalization of intervention
to subgroups, and adaptively follow studies until reaching a clear
clinical conclusion. Software and practical recommendations are provided
for developing adaptive monitoring study designs.

First, we provide novel extensions to Sequential Matched Randomization
which achieve greater covariate balance and efficiency than existing
Sequential Matched Randomization and traditional methods such as
Stratified Block Randomization. A dynamic and empirically-estimated
matching threshold allows all patients to match and relaxes an
assumption that baseline covariates are normally distributed. We allow
matches to break and rematch if a better matches enrolls in the study.
Under our method, randomization-based inference achieves nearly the same
efficiency as fitting an adjusted linear model to adjust for baseline
covariates. And, with greater covariate balance, an investigator may
better investigate the response of subgroups to intervention.

Second, we introduce adaptive monitoring with the second-generation
p-value which allows for following a study until ruling out effects
deemed trivial or until ruling out effects highly actionable to change
clinical practice. This grounds the clinical trial not only on
statistical significance but also clinical relevance, and can help
reduce the risk of a trial ending with inconclusive findings. We provide
a case study through the REACH, a Vanderbilt University Clinical Trial,
aimed to help patients with diabetes increase glycemic control and
better adhere to medications.

Finally, we develop statistical software and provide practical
recommendations for designing an adaptive monitoring trial using the
second generation p-value. The R package sgpvAM simulates data to
estimate operating characteristics for these trials; the required
simulation may be a barrier of entry to use this adaptive trial design.
We recommend a wait time before applying monitoring rules to control the
classical Type I error. When outcomes are not immediately observed
relative to enrollment, we recommend increasing the number of
observations before affirming a stopping alert. This reduces the risk of
stopping based on observed observations then finding a study
inconclusive when observing the remaining observations.

\hypertarget{references}{%
\chapter*{REFERENCES}\label{references}}
\addcontentsline{toc}{chapter}{REFERENCES}

\hypertarget{refs}{}
\leavevmode\hypertarget{ref-Armitage:1982ji}{}%
Armitage, Peter. 1982. ``The role of randomization in clinical trials.''
\emph{Statistics in Medicine} 1 (4): 345--52.

\leavevmode\hypertarget{ref-Atkinson:1982kt}{}%
Atkinson, A C. 1982. ``Optimum Biased Coin Designs for Sequential
Clinical Trials with Prognostic factors.'' \emph{Biometrika} 69 (1): 61.

\leavevmode\hypertarget{ref-Berchialla:2018bk}{}%
Berchialla, Paola, Dario Gregori, and Ileana Baldi. 2018. ``The Role of
Randomization in Bayesian and Frequentist Design of Clinical Trial.''
\emph{Topoi} 76 (4): 479.

\leavevmode\hypertarget{ref-Berger:2003im}{}%
Berger, Vance W, Anastasia Ivanova, and Maria Deloria Knoll. 2003.
``Minimizing predictability while retaining balance through the use of
less restrictive randomization procedures.'' \emph{Statistics in
Medicine} 22 (19): 3017--28.

\leavevmode\hypertarget{ref-Berry:2010fr}{}%
Berry, Scott, Bradley Carlin, J Lee, and Peter Müller. 2010. ``Bayesian
Adaptive Methods for Clinical Trials.'' CRC Press.

\leavevmode\hypertarget{ref-Blume:SGPV}{}%
Blume, Jeffrey D, Lucy DAgostino McGowan, William D Dupont, and Robert A
Greevy Jr. 2018. ``Second-generation p-values: Improved rigor,
reproducibility, \& transparency in statistical analyses.'' \emph{PloS
One} 13 (3): e0188299 EP.

\leavevmode\hypertarget{ref-Bothwell:2016bc}{}%
Bothwell, Laura E, Jeremy A Greene, Scott H Podolsky, and David S Jones.
2016. ``Assessing the Gold Standard--Lessons from the History of RCTs.''
\emph{New England Journal of Medicine} 374 (22): 2175--81.

\leavevmode\hypertarget{ref-Broglio:2014fr}{}%
Broglio, Kristine R, Jason T Connor, and Scott M Berry. 2014. ``Not too
big, not too small: a goldilocks approach to sample size selection.''
\emph{Journal of Biopharmaceutical Statistics} 24 (3): 685--705.

\leavevmode\hypertarget{ref-Ciolino:2011ff}{}%
Ciolino, Jody, Wenle Zhao, Renee Martin, and Yuko Palesch. 2011.
``Quantifying the cost in power of ignoring continuous covariate
imbalances in clinical trial randomization.'' \emph{Contemporary
Clinical Trials} 32 (2): 250--59.

\leavevmode\hypertarget{ref-DienerWest:1989uq}{}%
Diener-West, M, T W Dobbins, T L Phillips, and D F Nelson. 1989.
``Identification of an optimal subgroup for treatment evaluation of
patients with brain metastases using RTOG study 7916.''
\emph{International Journal of Radiation Oncology, Biology, Physics} 16
(3): 669--73.

\leavevmode\hypertarget{ref-fisher1935design}{}%
Fisher, Ronald A. 1935. ``The design of experiments. 1935.''
\emph{Oliver and Boyd, Edinburgh}.

\leavevmode\hypertarget{ref-Freedman:2008eq}{}%
Freedman, David A. 2008a. ``On regression adjustments to experimental
data.'' \emph{Advances in Applied Mathematics} 40 (2): 180--93.

\leavevmode\hypertarget{ref-Freedman:2008em}{}%
---------. 2008b. ``On regression adjustments in experiments with
several treatments.'' \emph{The Annals of Applied Statistics} 2 (1):
176--96.

\leavevmode\hypertarget{ref-Freedman:1984wz}{}%
Freedman, L S, D Lowe, and P Macaskill. 1984. ``Stopping rules for
clinical trials incorporating clinical opinion.'' \emph{Biometrics} 40
(3): 575--86.

\leavevmode\hypertarget{ref-Fu:2016jy}{}%
Fu, Haoda, Jin Zhou, and Douglas E Faries. 2016. ``Estimating optimal
treatment regimes via subgroup identification in randomized control
trials and observational studies.'' \emph{Statistics in Medicine} 35
(19): 3285--3302.

\leavevmode\hypertarget{ref-Greevy:2004ke}{}%
Greevy, R, B Lu, J H Silber, and P Rosenbaum. 2004. ``Optimal
multivariate matching before randomization.'' \emph{Biostatistics
(Oxford, England)}.

\leavevmode\hypertarget{ref-GreevyJr:2012hp}{}%
Greevy Jr., Robert A, Carlos G Grijalva, Christianne L Roumie, Cole
Beck, Adriana M Hung, Harvey J Murff, Xulei Liu, and Marie R Griffin.
2012. ``Reweighted Mahalanobis distance matching for cluster-randomized
trials with missing data.'' \emph{Pharmacoepidemiology and Drug Safety}
21 (May): 148--54.

\leavevmode\hypertarget{ref-Hill:1952hc}{}%
Hill, A Bradford. 1952. ``The Clinical Trial.'' \emph{New England
Journal of Medicine} 247 (4): 113--19.

\leavevmode\hypertarget{ref-Hobbs:2008ce}{}%
Hobbs, Brian P, and Bradley P Carlin. 2008. ``Practical Bayesian design
and analysis for drug and device clinical trials.'' \emph{Journal of
Biopharmaceutical Statistics} 18 (1): 54--80.

\leavevmode\hypertarget{ref-Kallus:2018um}{}%
Kallus, N, and 2018. 2018. ``Optimal a priori balance in the design of
controlled experiments.'' \emph{Journal of the Royal Statistical Society
Series B} 80: 85--112.

\leavevmode\hypertarget{ref-Kapelner:2014cu}{}%
Kapelner, Adam, and Abba Krieger. 2014. ``Matching on-the-fly:
Sequential allocation with higher power and efficiency.''
\emph{Biometrics} 70 (2): 378--88.

\leavevmode\hypertarget{ref-Kruschke:2013jy}{}%
Kruschke, John K. 2013. ``Bayesian estimation supersedes the t test.''
\emph{Journal of Experimental Psychology. General} 142 (2): 573--603.

\leavevmode\hypertarget{ref-Kruschke:2018bz}{}%
---------. 2018. ``Rejecting or Accepting Parameter Values in Bayesian
Estimation:'' \emph{Advances in Methods and Practices in Psychological
Science} 1 (2): 270--80.

\leavevmode\hypertarget{ref-LeylandJones:2003kt}{}%
Leyland-Jones, B. 2003. \emph{Breast cancer trial with erythropoietin
terminated unexpectedly}. The lancet oncology.

\leavevmode\hypertarget{ref-Lin:2013jh}{}%
Lin, Winston. 2013. ``Agnostic notes on regression adjustments to
experimental data: Reexamining Freedman's critique.'' \emph{The Annals
of Applied Statistics} 7 (1): 295--318.

\leavevmode\hypertarget{ref-Loux:2014bu}{}%
Loux, Travis M. 2014. ``Randomization, matching, and propensity scores
in the design and analysis of experimental studies with measured
baseline covariates.'' \emph{Statistics in Medicine} 34 (4): 558--70.

\leavevmode\hypertarget{ref-Atkinson:1999hq}{}%
Medicine, AC Atkinson Statistics in, and 1999. 1999. ``Optimum
biased-coin designs for sequential treatment allocation with covariate
information (With Discussion).'' \emph{Journal of the Royal Statistical
Society Series B} 18 (14): 1753--5.

\leavevmode\hypertarget{ref-Morgan:2012iq}{}%
Morgan, K L, and D B Rubin. 2012. ``Rerandomization to improve covariate
balance in experiments.'' \emph{The Annals of Statistics}.

\leavevmode\hypertarget{ref-Nelson:2018bw}{}%
Nelson, Lyndsay A, Kenneth A Wallston, Sunil Kripalani, Robert A Greevy
Jr., Tom A Elasy, Erin M Bergner, Chad K Gentry, and Lindsay S Mayberry.
2018. ``Mobile Phone Support for Diabetes Self-Care Among Diverse
Adults: Protocol for a Three-Arm Randomized Controlled Trial.''
\emph{JMIR Research Protocols} 7 (4): e92.

\leavevmode\hypertarget{ref-Peirce:1884wra}{}%
Peirce, C S, and J Jastrow. 1884. ``On small differences in sensation.''
\emph{Memoirs of the National Academy of Sciences} III: 73--83.

\leavevmode\hypertarget{ref-Pocock:1975wd}{}%
Pocock, Stuart J, and Richard Simon. 1975. ``Sequential Treatment
Assignment with Balancing for Prognostic Factors in the Controlled
Clinical Trial.'' \emph{Biometrics} 31 (1): 103--15.

\leavevmode\hypertarget{ref-Pocock:2016ey}{}%
Pocock, Stuart J, and Gregg W Stone. 2016a. ``The Primary Outcome Fails
- What Next?'' \emph{New England Journal of Medicine} 375 (9): 861--70.

\leavevmode\hypertarget{ref-Pocock:2016ca}{}%
---------. 2016b. ``The Primary Outcome Is Positive - Is That Good
Enough?'' \emph{New England Journal of Medicine} 375 (10): 971--79.

\leavevmode\hypertarget{ref-Rosenberger:2008cm}{}%
Rosenberger, W F, and O Sverdlov. 2008. ``Handling covariates in the
design of clinical trials.''

\leavevmode\hypertarget{ref-Senn:2010bg}{}%
Senn, Stephen, Vladimir V Anisimov, and Valerii V Fedorov. 2010.
``Comparisons of minimization and Atkinson\&apos;s algorithm.''
\emph{Statistics in Medicine} 29 (7-8): 721--30.

\leavevmode\hypertarget{ref-Spiegelhalter:1994cn}{}%
Spiegelhalter, David J, Laurence S Freedman, and Mahesh K B Parmar.
1994. ``Bayesian Approaches to Randomized Trials.'' \emph{Journal of the
Royal Statistical Society: Series A (Statistics in Society)} 157 (3):
357.

\leavevmode\hypertarget{ref-Taves:1974hn}{}%
Taves, Donald R. 1974. ``Minimization: A new method of assigning
patients to treatment and control groups.'' \emph{Clinical Pharmacology
\&Amp; Therapeutics} 15 (5): 443--53.

\leavevmode\hypertarget{ref-Wei:1988if}{}%
Wei, L J, and John M Lachin. 1988. ``Properties of the urn randomization
in clinical trials.'' \emph{Controlled Clinical Trials} 9 (4): 345--64.

\leavevmode\hypertarget{ref-Quan:SeqRerand}{}%
Zhou, Quan, Philip Ernst, Kari Lock Morgan, Donald B Rubin, and Anru
Zhang. 2018. ``Sequential Rerandomization.'' \emph{arXiv.org}, April,
1--23.

\end{document}
