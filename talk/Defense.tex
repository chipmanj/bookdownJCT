%% LyX 2.2.1 created this file.  For more info, see http://www.lyx.org/.
%% Do not edit unless you really know what you are doing.
\documentclass[10pt]{beamer}
\usepackage[T1]{fontenc}
\usepackage{fancyvrb}
\setcounter{secnumdepth}{3}
\setcounter{tocdepth}{3}
\usepackage{url}
\usepackage{animate}
\usepackage{natbib}     
\usepackage{bibentry}
\usepackage{relsize} 
\usepackage{pgfpages}
\usepackage{amsmath}% http://ctan.org/pkg/amsmath
\usepackage{bm}
\usepackage{blkarray}
\usepackage{mathtools}
\usepackage{hyperref}
\usepackage{graphicx}
\usepackage[export]{adjustbox}

\usepackage{booktabs}
\usepackage{longtable}
\usepackage{array}
\usepackage{multirow}
\usepackage{wrapfig}
\usepackage{float}
\usepackage{colortbl}
\usepackage{pdflscape}
\usepackage{tabu}
\usepackage{threeparttable}
\usepackage{threeparttablex}
\usepackage[normalem]{ulem}
\usepackage{makecell}
\usepackage{xcolor}


\renewcommand\theadalign{bc}
% \renewcommand\theadfont{\bfseries}
\renewcommand\theadgape{\Gape[4pt]}
\renewcommand\cellgape{\Gape[4pt]}



\ifx\hypersetup\undefined
  \AtBeginDocument{%
    \hypersetup{unicode=true,pdfusetitle,
 bookmarks=true,bookmarksnumbered=false,bookmarksopen=false,
 breaklinks=false,pdfborder={0 0 0},pdfborderstyle={},backref=false,colorlinks=false}
  }
\else
  \hypersetup{unicode=true,pdfusetitle,
 bookmarks=true,bookmarksnumbered=false,bookmarksopen=false,
 breaklinks=false,pdfborder={0 0 0},pdfborderstyle={},backref=false,colorlinks=false}
\fi
\usepackage{breakurl}
\usepackage{subcaption}

%\usepackage[usenames]{colors} 
%\definecolor{darkred}{rgb}{0.545,0,0} 
%\definecolor{midnightblue}{rgb}{0.098,0.098,0.439} 
% \DefineVerbatimEnvironment{Sinput}{Verbatim}{fontshape=sl,formatcom={\color{blue}}} 
% \DefineVerbatimEnvironment{Soutput}{Verbatim}{formatcom={\color{red}}} 
% \DefineVerbatimEnvironment{Scode}{Verbatim}{fontshape=sl,formatcom={\color{blue}}} 

\DefineVerbatimEnvironment{Sinput}{Verbatim} {xleftmargin=2em, formatcom={\color{red}}}
\DefineVerbatimEnvironment{Soutput}{Verbatim}{xleftmargin=2em, formatcom={\color{red}}}
\DefineVerbatimEnvironment{Scode}{Verbatim}{xleftmargin=2em, formatcom={\color{red}}}
%\fvset{listparameters={\setlength{\topsep}{0pt}}}
%\renewenvironment{Schunk}{\vspace{\topsep}}{\vspace{\topsep}}

% \newcommand{\figdur}[1]{/Users/chipmanj/Dropbox/statistics/work/vandy/RandomizationMethods/DFCI/developtalk/#1/}

% \graphicspath{{\figdur{visualPresentations}},{\figdur}}

\graphicspath{{\string~/Dropbox/statistics/work/vandy/RandomizationMethods/DFCI/developtalk/}}


\newcommand{\ft}[1]{\frametitle{#1}}
\newcommand{\bi}{\begin{itemize}}
\newcommand{\ei}{\end{itemize}}
%\newcommand{\ba}{\begin{align*}}
%\newcommand{\ea}{\end{align*}}
\newcommand{\specialcell}[2][c]{%
  \begin{tabular}[#1]{@{}c@{}}#2\end{tabular}}

\newenvironment{myitemize}{%
   \setlength{\topsep}{0pt}
   \setlength{\partopsep}{0pt}
   \renewcommand*{\@listi}{\leftmargin\leftmargini \parsep\z@ \topsep\z@ \itemsep\z@}
   \let\@listI\@listi
   \itemize
}{\enditemize}


%Macros to make graphics insertions easy
%Command for sizing to width    \figw{file}{fraction of \textwidth}
\newcommand{\figw}[2]{\centerline{\includegraphics[width=#2\textwidth]{#1}}}
%Command for sizing to height   \figh{file}{fraction of \textheight}
\newcommand{\figh}[2]{\centerline{\includegraphics[height=#2\textheight]{#1}}}
%Use \figh{graphics file name}{1} to size to whole text height
%For graphics needing no shrinkage:  \fig{file}
\newcommand{\fig}[1]{\centerline{\includegraphics{#1}}}

\setbeamertemplate{caption}{\raggedright\insertcaption\par}


\addtobeamertemplate{navigation symbols}{}{%
    \usebeamerfont{footline}%
    \usebeamercolor[fg]{footline}%
    \hspace{1em}%
    \insertframenumber/\inserttotalframenumber
}
\setbeamercolor{footline}{fg=black}

% \DefineVerbatimEnvironment{Sinput}{Verbatim}{fontseries=bc,frame=single}
% \DefineVerbatimEnvironment{Soutput}{Verbatim}{fontseries=bc,frame=leftline}
% \DefineVerbatimEnvironment{Scode}{Verbatim}{fontseries=bc}
% 
% \usepackage{minted}
% 
% \renewenvironment{Sinput}{\minted[frame=single]{r}}{\endminted}
% \DefineVerbatimEnvironment{Soutput}{Verbatim}{frame=leftline}
% \DefineVerbatimEnvironment{Scode}{Verbatim}{}


\makeatletter

%%%%%%%%%%%%%%%%%%%%%%%%%%%%%% Textclass specific LaTeX commands.
 % this default might be overridden by plain title style
 \newcommand\makebeamertitle{\frame{\maketitle}}%
 % (ERT) argument for the TOC
 \AtBeginDocument{%
   \let\origtableofcontents=\tableofcontents
   \def\tableofcontents{\@ifnextchar[{\origtableofcontents}{\gobbletableofcontents}}
   \def\gobbletableofcontents#1{\origtableofcontents}
 }

%%%%%%%%%%%%%%%%%%%%%%%%%%%%%% User specified LaTeX commands.
\usetheme{vandy}

\makeatother

\usepackage{Sweave}
\begin{document}
\Sconcordance{concordance:Defense.tex:Defense.Rnw:%
1 143 1 1 0 3 1 1 5 1 1 1 5 1 1 1 15 482 1}

\nobibliography*









\title[RCT Efficiency, Balance, and EFficacy]{Sequential Rematching Randomization and Adaptive Monitoring with the Second-Generation p-value to increase the efficiency and efficacy of Randomized Clinical Trials}
\author[Jonathan Chipman]{Jonathan Chipman, MS\\[3mm]Department of Biostatistics\\[1mm]Vanderbilt University}
\date{May 10, 2019}

\makebeamertitle


\begin{frame}{RCT Costs and Benefits}

\vspace{0.2cm}

2015-2016 Pivotal Trials accepted by FDA \cite{Moore:wc}
\bi
  \item 138 Trials to approve 59 novel therapeutic drugs
  \item Enrolled on median 488 patients (IQR: 230, 740)
  \item Cost per patient on median \$41K (IQR: \$32K - \$82K)
\ei

% NINDS Reviewed costs of trials between 1977 and 2002
% \bi
%   \item Overall quality of life benefits $>>$ Overall costs
% \ei


\end{frame}



\begin{frame}{Outline}

Three papers to increase study efficiency and efficacy.

\vspace{0.2cm}
\bi
  \item Paper 1: Sequential Rematched Randomization
  \item Paper 2: Adaptive Monitoring Using the Second Generation p-value
  \item Paper 3: Estimation of operating characteristics of AM with SGPV
\ei


Motivating Example: REACH Trial



\end{frame}



\section{REACH RCT}

\begin{frame}[fragile]{REACH Trial}


\begin{centering}
\vspace{0.3cm}

\textbf{R}apid \textbf{E}ducation/Encouragement \textbf{A}nd \textbf{C}ommunications for \textbf{H}ealth

\vspace{0.6cm}

\bi
	{\setlength\itemindent{50pt}\item [Population:] Adults with Type 2 Diabetes (DM)}
	{\setlength\itemindent{50pt}\item [Purpose:] Increase glycemic control and adherence to medications}
	{\setlength\itemindent{50pt}\item [Main Intervention:] Text message-delivered diabetes support for 12 months}
	{\setlength\itemindent{50pt}\item [Outcome:] Glycemic control (A1c) compared to control}
	{\setlength\itemindent{50pt}\item [Multi-site enrollment:] 512 patients from Vanderbilt and Non-Vanderbilt Clinics}
	{\setlength\itemindent{50pt}\item [Clinical Relevance:] Change clinical practice -- Decrease in A1c of 0.5 or more}
\ei

\vspace{0.4cm}

% /Users/Chipman/Dropbox/statistics/work/byu/2018 Faculty application/slides

\figw{\string~/Dropbox/statistics/work/vandy/RandomizationMethods/ExtendingMOTF/FigsRefs/visualPresentations/REACHphone}{0.7}

\end{centering}

\end{frame}


\begin{frame}{Key Baseline Covariates}


\begin{columns}[t]
  \begin{column}{0.5\textwidth}
    Biological Factors
    \bi
      \item Baseline A1c$^{\ast}$
      \item Age at baseline
      \item Race $/$ Ethnicity
      \item DM Rx type$^{\ast}$
      \item Time since DM dx$^{\ast}$
    \ei
  \end{column}
  \begin{column}{0.5\textwidth}
    Socio-economic Factors
    \bi
      \item Yrs of education
      \item Income level
      \item Insurance type
    \ei
  \end{column}
\end{columns}

\vspace{0.3cm}
$^{\ast}$ Post-hoc analysis: Greater association with outcome

\end{frame}


% \begin{frame}{Stratified Block Randomization}
% 
% \bi
%   \item Randomize within categories of similar patients (ex: Site)
%   \item Continuous covariates must be categorized
%   \item REACH Example: Randomize within site and Baseline A1c (6 strata)
% \ei
% 
% \vspace{0.3cm}
% \figw{\string~/Dropbox/statistics/work/vandy/RandomizationMethods/ExtendingMOTF/FigsRefs/visualPresentations/stratification/Slide1}{.5}
% 
% \end{frame}


\begin{frame}{Stratified Block Randomization}

\bi
  \item Limited in number of covariates creating strata
  \item 4000+ Strata - Stratifying on all covariates$^{\ast}$
  \item 2 * 3 * 3 * 2 = 36 Strata - Stratifying on:
    \bi
      \item 2 Sites
      \item 3 Baseline A1c levels
      \item 3 DM Rx Types
      \item 2 Time since DM Dx levels
    \ei
\ei


% \vspace{0.3cm}
% $^{\ast}$Continuous baseline covariates categorized as:
% 
% \begin{table}[]
% \begin{flushleft}
% \begin{tabular}{@{}llll@{}}
% Baseline A1c:            & \textless{}\textcolor{white}{0}7 & 7-8    &$\ge$8 \\
% Age:                     & \textless{}60 & $\ge$60 &            \\
% Yrs of Education:        & \textless{}12 & $\ge$12 &            \\
% Time since DM Dx: & \textless{}10 & $\ge$10 &
% \end{tabular}
% \end{flushleft}
% \end{table}

\end{frame}


\begin{frame}{Matched Randomization \citep{Greevy:2004ke}}
\addtocounter{framenumber}{-1}


\figw{\string~/Dropbox/statistics/work/vandy/RandomizationMethods/ExtendingMOTF/FigsRefs/visualPresentations/BipartiteImage}{.5}

Refined stratification into pairs of patients using distance matrix
\bi
\item Pairs selected for smallest total distance
\item Allows for any number of continuous and categorical covariates
\item All participants known before randomization
\ei

\end{frame}


\frame{
\ft{Sequential Matching \citep{Kapelner:2014cu}}
\centering
\vspace*{\fill}
\animategraphics[autoplay,height=5cm]{2}{Anime}{0}{0}
\vspace*{\fill}
}


\frame{
\ft{Sequential Matching \citep{Kapelner:2014cu}}
\addtocounter{framenumber}{-1}
\centering
\vspace*{\fill}
\animategraphics[autoplay,height=5cm]{2}{Anime}{1}{10}
\vspace*{\fill}
}

\frame{
\ft{Sequential Matching \citep{Kapelner:2014cu}}
\addtocounter{framenumber}{-1}
\centering
\vspace*{\fill}
\animategraphics[autoplay,height=5cm]{.7}{Anime}{14}{18}
\vspace*{\fill}
}

\frame{
\ft{Sequential Matching \citep{Kapelner:2014cu}}
\addtocounter{framenumber}{-1}
\centering
\vspace*{\fill}
\animategraphics[autoplay,height=5cm]{.5}{Anime}{19}{21}
\vspace*{\fill}
}

\frame{
\ft{Sequential Matching \citep{Kapelner:2014cu}}
\addtocounter{framenumber}{-1}
\centering
\vspace*{\fill}
\animategraphics[autoplay,height=5cm]{.5}{Anime}{22}{24}
\vspace*{\fill}
}



\frame{
\ft{Matching OTF, some additional notes \citep{Kapelner:2014cu}}
Pre-specification:
\bi
\item Initial reservoir size
\item Threshold to denote degree of similarity
\ei
\vspace{\baselineskip}


Fixed Threshold: Match is better than [20\%] of random matches
\bi
  \item Mahalanobis Distance of random pairs scales to $F_{(p,n-p)}$
  \item Assumes normally distributed baseline covariates
\ei

\vspace{\baselineskip}
Reservoir:
\bi
\item Not required to deplete
\item May result in unequal treatment group sizes
\ei
\vspace{\baselineskip}

}


\frame{

\ft{Extension 1: Dynamic Threshold}

Dynamic percentile of random match distances

\bi
\item[1.] Chance of matching now versus later
\item[2.] Random match distances (from data, ie not assumed normal)
\item[3.] Remove threshold at end to ensure all patients match
\ei

Patients enroll in $b$ enrollment blocks\\
$|| U_b ||$: Number of unmatched patients at b$^{th}$ enrollment\\
$|| R_b ||$: Number remaining patients after b$^{th}$ enrollment\\
\begin{equation*}
Q_b = \frac{||  U_b || - 1}{ ||U_b|| + ||R_b|| - 1}
\end{equation*}

\begin{equation*}
Threshold_b = \begin{cases} F_b^{-1} (Q_b) &  || U_b || < || R_b || \\
                    \text{best match(es)}      &  || U_b || \ge || R_b || \\
      \end{cases}
\end{equation*}

$F_b$ is estimated by $\hat{F_b}$ by bootstrap sampling random matches from distance matrix.

}


\frame{
\ft{Extension 2: Sequential Re-Matching}

Sequential Matching
\bi
\item Formed matches remain through study
% \item TIM: Treatment Independent Matching
\ei
\vspace*{\fill}

Sequential Re-Matching
\bi
  \item Matches are allowed to break if a better match enters
  \item However, participants keep original treatment assignment
  \bi
    \item New participant: Match to anybody
    \item Former enrolled participant: Match to anyone of opposite treatment or new participant
  \ei
  % \item TDM: Treatment Dependent Matching
\ei
}


\begin{frame}[fragile]{ }

\centering
Case study using REACH Data to \\
comparing randomization schemes

\vspace{0.3cm}

\bi
	{\setlength\itemindent{100pt}\item[Balance] of baseline covariates}
  {\setlength\itemindent{100pt}\item[Efficiency] estimating treatment effect}
\ei

\vspace{0.5cm}
Results will vary based on baseline covariates
\bi
  {\setlength\itemindent{80pt}\item Number of patients}
  {\setlength\itemindent{80pt}\item Number of covariates}
  {\setlength\itemindent{80pt}\item Distribution of covariates}
  {\setlength\itemindent{80pt}\item Adjusted $R^2$ of covariates}
\ei

\end{frame}


\begin{frame}{Simulations to evaluate Sequential Rematching Randomization}

% \vspace{0.2cm}

Generate 20K MCMC replicates under potential outcomes framework:

\bi
  \item Bootstrap sample 512 observations from REACH trial
  \item Outcomes
  \bi
    \item Y(0) = Predicted three month A1c + random residual
    \item Y(1) = Y(0) - 0.5
  \ei
  \item Generate randomization scheme under
  \bi
    \item Block Randomization
    \item Stratified Block Randomization
    \item Matched Randomization
  \ei
  \item Observe
  \bi
    \item Balance: Average Standardized Mean Difference among all covariates
    \item Efficiency: CI Width of 1) fully-adjusted linear model and 2) Permutation distribution
  \ei
\ei


Where Stratified and Matched randomization adjusts for
  \bi
    \item Site, Baseline A1c, DM Rx Type, and Time since DM Dx
    \item All baseline covariates
  \ei

\end{frame}



% \begin{frame}[fragile]{ }
% \figw{\string~/Dropbox/statistics/work/vandy/RandomizationMethods/ExtendingMOTF/FigsRefs/visualPresentations/MaxSMD_defense}{.825}
% \end{frame}

\begin{frame}[fragile]{ }
\figw{\string~/Dropbox/statistics/work/vandy/RandomizationMethods/ExtendingMOTF/FigsRefs/visualPresentations/MaxSMD_defense}{.825}
\end{frame}



\begin{frame}[fragile]{ }
\figw{\string~/Dropbox/statistics/work/vandy/RandomizationMethods/ExtendingMOTF/FigsRefs/visualPresentations/modelAdjCIwidth_defense}{.825}
\end{frame}

% \begin{frame}[fragile]{ }
% \figw{\string~/Dropbox/statistics/work/vandy/RandomizationMethods/ExtendingMOTF/FigsRefs/visualPresentations/permutationCIwidth_defense}{.825}
% \end{frame}

\begin{frame}[fragile]{ }
\figw{\string~/Dropbox/statistics/work/vandy/RandomizationMethods/ExtendingMOTF/FigsRefs/visualPresentations/permutationCIwidth_defense}{.825}
\end{frame}


\begin{frame}{Recommendations regarding balance and efficiency}

Conclusions
\bi
  \item Greatest overall balance by adjusting for all baseline covariates
  \vspace{0.5cm}
  \item Greatest linear gains in efficiency
    \bi
      \item Model efficiency: adjusting randomization to all baseline covariates
      \item Permutation efficiency: adjusting randomization to priority covariates
    \ei
  \vspace{0.5cm}
  \item Permutation efficiency achieves nearly same efficiency as fully-adjusted model with Block Randomization
\ei

\vspace{0.5cm}
Practical Recommendations
\bi
  \item Recommend adjusting to all baseline covariates
  \item Adaptively monitor until reaching a clear clinical conclusion
\ei

\end{frame}


\begin{frame}
Paper 2: Adaptive Monitoring Using the Second Generation p-value

\end{frame}




\begin{frame}{Prematurely Ending Clinical Trial(s)}

\vspace{.75cm}

\center{Towards a Revolution in COPD Health (TORCH) (\cite{Calverley:2007gx})

\vspace{0.5cm}

Primary Aim: Establish whether beta-agonist (salmeterol plus fluticasone propionate) has survival benefit in patients with chronic obstructive pulmonary disease}

\vspace{.5cm}

\bi
\item[2007] 6112 patients
\item HR 0.825 (95\% CI: 0.681-1.002, p-adjusted=0.052)
\item Awkward Conclusion: primary outcome did not reach statistical significance, yet 'significant benefits in all other outcomes.'
\ei


%    \begin{textblock}{50}(80,87)
% \cite{Pocock:2016ey}
%     \end{textblock}

\end{frame}



\begin{frame}{Prematurely Ending Clinical Trial(s)}
\addtocounter{framenumber}{-1}

\vspace{.75cm}

\center{Towards a Revolution in COPD Health (TORCH) (\cite{Calverley:2007gx})

\vspace{0.5cm}

Primary Aim: Establish whether beta-agonist (salmeterol plus fluticasone propionate) has survival benefit in patients with chronic obstructive pulmonary disease}

\vspace{.5cm}

\bi
\item[2007] 6112 patients
\item HR 0.825 (95\% CI: 0.681-\textcolor{red}{1.000}, p-adjusted=\textcolor{red}{0.05})
\item Awkward Conclusion: primary outcome did not reach statistical significance, yet 'significant benefits in all other outcomes.'
\ei


\end{frame}



\begin{frame}{Prematurely Ending Clinical Trial(s)}
\addtocounter{framenumber}{-1}

\vspace{.75cm}

\center{Towards a Revolution in COPD Health (TORCH) (\cite{Calverley:2007gx})

\vspace{0.5cm}

Primary Aim: Establish whether beta-agonist (salmeterol plus fluticasone propionate) has survival benefit in patients with chronic obstructive pulmonary disease}

\vspace{.5cm}

\bi
\item[2007] 6112 patients
\item HR 0.825 (95\% CI: 0.681-\textcolor{red}{0.998}, p-adjusted=\textcolor{red}{0.0498})
\item Awkward Conclusion: primary outcome did not reach statistical significance, yet 'significant benefits in all other outcomes.'
\ei



\end{frame}




\begin{frame}{Second Generation p-value (SGPV; \cite{Blume:SGPV})}

\begin{center} What proportion of interval overlaps with trivial effects? \end{center}

\figh{\string~/Dropbox/statistics/work/vandy/AdaptiveMonitoring/Conferences/SCT/slides/SGPVv02}{.3}

\vspace{0.5cm}

Interpretation of SGPV$_T$ (T for trivial effects)
\bi
\item SGPV$_T$ = 0.0: Evidence to rule out trivial effects
\item SGPV$_T$ = 1.0: Evidence to rule out non-trivial effects
\item SGPV$_T$ = 0.5: Inconclusive, need more data
\ei


%    \begin{textblock}{50}(80,87)
% 	   	See also, \cite{Kruschke:2013jya}
%     \end{textblock}


\end{frame}




\begin{frame}{Second Generation p-value (SGPV; \cite{Blume:SGPV})}


\begin{centering}
\vspace{0.3cm}

Example of indifferenze zone with TORCH

\vspace{1.3cm}


\figw{\string~/Dropbox/statistics/work/vandy/AdaptiveMonitoring/Conferences/SCT/slides/izTORCH}{.6}

\vspace{0.6cm}

If HR were 0.825 (95\% CI: 0.681-0.899)\\

Interpretation of SGPV$_T$ = 0: Rule out trivial effects

\vspace{0.5cm}


\end{centering}



\end{frame}



\begin{frame}[fragile]{REACH}


\begin{centering}
\vspace{0.3cm}

\textbf{R}apid \textbf{E}ducation/Encouragement \textbf{A}nd \textbf{C}ommunications for \textbf{H}ealth

\vspace{1.3cm}


\figw{\string~/Dropbox/statistics/work/vandy/AdaptiveMonitoring/Conferences/SCT/slides/izREACH}{0.5}

\vspace{0.6cm}

A decrease is HbA1c of 0.15 (ex. 7.5 to 7.35)\\is considered essentially no change

\end{centering}

\end{frame}



\begin{frame}[fragile]{REACH}

\begin{centering}
\vspace{0.3cm}

\textbf{R}apid \textbf{E}ducation/Encouragement \textbf{A}nd \textbf{C}ommunications for \textbf{H}ealth

\vspace{1.3cm}


\figw{\string~/Dropbox/statistics/work/vandy/AdaptiveMonitoring/Conferences/SCT/slides/cmREACH}{0.5}

\vspace{0.6cm}

A decrease is HbA1c of 0.5 (ex. 7.5 to 7.0)\\would highly suggest adopting intervention

\end{centering}


\end{frame}


\begin{frame}[fragile]{REACH}
\begin{centering}
\vspace{0.3cm}

\textbf{R}apid \textbf{E}ducation/Encouragement \textbf{A}nd \textbf{C}ommunications for \textbf{H}ealth

\vspace{1.3cm}


\figw{\string~/Dropbox/statistics/work/vandy/AdaptiveMonitoring/Conferences/SCT/slides/merrREACH}{0.5}

\vspace{0.6cm}

A decrease is HbA1c of between 0.15 and 0.5 is interesting\\it may suggest adopting intervention


\end{centering}


\end{frame}





\begin{frame}[fragile]{REACH}


\begin{centering}
\vspace{0.3cm}

\textbf{R}apid \textbf{E}ducation/Encouragement \textbf{A}nd \textbf{C}ommunications for \textbf{H}ealth

\vspace{1.3cm}


\figw{\string~/Dropbox/statistics/work/vandy/AdaptiveMonitoring/Conferences/SCT/slides/izCmREACH}{0.5}

\vspace{0.6cm}

SGPV$_T$ to rule out trivial effects\\
SGPV$_{HA}$ to rule out highly-actionable effects

\vspace{0.6cm}

\bi
	{\setlength\itemindent{15pt}\item[] SGPV$_{T}$ = 0.0: Ruled out trivial effects}
	{\setlength\itemindent{15pt}\item[] SGPV$_{HA}$ = 0.0: Ruled out highly actionable effects}
\ei

\end{centering}


\end{frame}





\begin{frame}{Adaptive Monitoring with SGPV}
\addtocounter{framenumber}{-1}

\vspace{.3cm}

\bi
	{\setlength\itemindent{15pt}\item[Wait] interval width stabilizes}
	{\setlength\itemindent{15pt}\item[Monitor] interval and SGPV at desired looks}
	{\setlength\itemindent{15pt}\item[Alert] SGPV$_{T}$ = 0.0: Ruled out trivial effects}
	{\setlength\itemindent{15pt}\item[] SGPV$_{HA}$ = 0.0: Ruled out highly actionable effects}
	{\setlength\itemindent{15pt}\item[Affirm] stop if same conclusion [40] patients later}
	{\setlength\itemindent{15pt}\item[] end of resources (n = 512)}
	{\setlength\itemindent{15pt}\item[Report] only the final interval when stopping}
\ei


\end{frame}


\graphicspath{{\string~/Dropbox/statistics/work/vandy/AdaptiveMonitoring/Conferences/SCT/slides/}}

\begin{frame}{Adaptive Monitoring with SGPV}
\addtocounter{framenumber}{-1}

\vspace{.3cm}

\bi
  {\setlength\itemindent{15pt}\item[Wait] interval width stabilizes}
	{\setlength\itemindent{15pt}\item[Monitor] interval and SGPV at desired looks}
	{\setlength\itemindent{15pt}\item[Alert] SGPV$_{T}$ = 0.0: Ruled out trivial effects}
	{\setlength\itemindent{15pt}\item[] SGPV$_{HA}$ = 0.0: Ruled out highly actionable effects}
	{\setlength\itemindent{15pt}\item[Affirm] stop if same conclusion [40] patients later}
	{\setlength\itemindent{15pt}\item[] end of resources (n = 512)}
	{\setlength\itemindent{15pt}\item[Report] only the final interval when stopping}
\ei

% \begin{center}\animategraphics[autoplay,height=4cm,every=1]{65}{mu00exampleSeed7_}{1}{1}\end{center}


\end{frame}



\begin{frame}{Adaptive Monitoring with SGPV}
\addtocounter{framenumber}{-1}

\vspace{.3cm}

\bi
  {\setlength\itemindent{15pt}\item[Wait] interval width stabilizes}
	{\setlength\itemindent{15pt}\item[Monitor] interval and SGPV at desired looks}
	{\setlength\itemindent{15pt}\item[Alert] SGPV$_{T}$ = 0.0: Ruled out trivial effects}
	{\setlength\itemindent{15pt}\item[] SGPV$_{HA}$ = 0.0: Ruled out highly actionable effects}
	{\setlength\itemindent{15pt}\item[Affirm] stop if same conclusion [40] patients later}
	{\setlength\itemindent{15pt}\item[] end of resources (n = 512)}
	{\setlength\itemindent{15pt}\item[Report] only the final interval when stopping}
\ei

% \begin{center}\animategraphics[autoplay,height=4cm,every=5]{65}{\string~/Dropbox/statistics/work/vandy/AdaptiveMonitoring/Conferences/SCT/slides/mu00exampleSeed7_}{1}{301}\end{center}


\end{frame}



\begin{frame}{Adaptive Monitoring with SGPV}
\addtocounter{framenumber}{-1}

\vspace{.3cm}

\bi
  {\setlength\itemindent{15pt}\item[Wait] interval width stabilizes}
	{\setlength\itemindent{15pt}\item[Monitor] interval and SGPV at desired looks}
	{\setlength\itemindent{15pt}\item[Alert] SGPV$_{T}$ = 0.0: Ruled out trivial effects}
	{\setlength\itemindent{15pt}\item[] SGPV$_{HA}$ = 0.0: Ruled out highly actionable effects}
	{\setlength\itemindent{15pt}\item[Affirm] stop if same conclusion [40] patients later}
	{\setlength\itemindent{15pt}\item[] end of resources (n = 512)}
	{\setlength\itemindent{15pt}\item[Report] only the final interval when stopping}
\ei

% \begin{center}\animategraphics[autoplay,height=4cm,every=5]{65}{\string~/Dropbox/statistics/work/vandy/AdaptiveMonitoring/Conferences/SCT/slides/mu00example}{4}{484}\end{center}


\end{frame}


\begin{frame}{Adaptive Monitoring with SGPV}
\addtocounter{framenumber}{-1}

\vspace{.3cm}

\bi
  {\setlength\itemindent{15pt}\item[Wait] interval width stabilizes}
	{\setlength\itemindent{15pt}\item[Monitor] interval and SGPV at desired looks}
	{\setlength\itemindent{15pt}\item[Alert] SGPV$_{T}$ = 0.0: Ruled out trivial effects}
	{\setlength\itemindent{15pt}\item[] SGPV$_{HA}$ = 0.0: Ruled out highly actionable effects}
	{\setlength\itemindent{15pt}\item[Affirm] stop if same conclusion [40] patients later}
	{\setlength\itemindent{15pt}\item[] end of resources (n = 512)}
	{\setlength\itemindent{15pt}\item[Report] only the final interval when stopping}
\ei


% \begin{center}\animategraphics[autoplay,height=4cm,every=1]{65}{\string~/Dropbox/statistics/work/vandy/AdaptiveMonitoring/Conferences/SCT/slides/mu07example}{1}{1}\end{center}


\end{frame}


\begin{frame}{Adaptive Monitoring with SGPV}
\addtocounter{framenumber}{-1}

\vspace{.3cm}

\bi
  {\setlength\itemindent{15pt}\item[Wait] interval width stabilizes}
	{\setlength\itemindent{15pt}\item[Monitor] interval and SGPV at desired looks}
	{\setlength\itemindent{15pt}\item[Alert] SGPV$_{T}$ = 0.0: Ruled out trivial effects}
	{\setlength\itemindent{15pt}\item[] SGPV$_{HA}$ = 0.0: Ruled out highly actionable effects}
	{\setlength\itemindent{15pt}\item[Affirm] stop if same conclusion [40] patients later}
	{\setlength\itemindent{15pt}\item[] end of resources (n = 512)}
	{\setlength\itemindent{15pt}\item[Report] only the final interval when stopping}
\ei


% \begin{center}\animategraphics[autoplay,height=4cm,every=5]{65}{\string~/Dropbox/statistics/work/vandy/AdaptiveMonitoring/Conferences/SCT/slides/mu07example}{2}{257}\end{center}


\end{frame}




\begin{frame}{Adaptive Monitoring with SGPV}
\addtocounter{framenumber}{-1}

\vspace{.3cm}

\bi
  {\setlength\itemindent{15pt}\item[Wait] interval width stabilizes}
	{\setlength\itemindent{15pt}\item[Monitor] interval and SGPV at desired looks}
	{\setlength\itemindent{15pt}\item[Alert] SGPV$_{T}$ = 0.0: Ruled out trivial effects}
	{\setlength\itemindent{15pt}\item[] SGPV$_{HA}$ = 0.0: Ruled out highly actionable effects}
	{\setlength\itemindent{15pt}\item[Affirm] stop if same conclusion [40] patients later}
	{\setlength\itemindent{15pt}\item[] end of resources (n = 512)}
	{\setlength\itemindent{15pt}\item[Report] only the final interval when stopping}
\ei



% \begin{center}\animategraphics[autoplay,height=4cm,every=1]{65}{\string~/Dropbox/statistics/work/vandy/AdaptiveMonitoring/Conferences/SCT/slides/mu03example}{1}{1}\end{center}


\end{frame}

\begin{frame}{Adaptive Monitoring with SGPV}
\addtocounter{framenumber}{-1}

\vspace{.3cm}

\bi
  {\setlength\itemindent{15pt}\item[Wait] interval width stabilizes}
	{\setlength\itemindent{15pt}\item[Monitor] interval and SGPV at desired looks}
	{\setlength\itemindent{15pt}\item[Alert] SGPV$_{T}$ = 0.0: Ruled out trivial effects}
	{\setlength\itemindent{15pt}\item[] SGPV$_{HA}$ = 0.0: Ruled out meaningful effects}
	{\setlength\itemindent{15pt}\item[Affirm] stop if same conclusion [40] patients later}
	{\setlength\itemindent{15pt}\item[] end of resources (n = 512)}
	{\setlength\itemindent{15pt}\item[Report] only the final interval when stopping}
\ei



% \begin{center}\animategraphics[autoplay,height=4cm,every=5]{65}{\string~/Dropbox/statistics/work/vandy/AdaptiveMonitoring/Conferences/SCT/slides/mu03example}{4}{224}\end{center}


\end{frame}









\begin{frame}

	\figh{\string~/Dropbox/statistics/work/vandy/AdaptiveMonitoring/manuscript/figs/powerCalibratedLooks20}{.9}

\end{frame}


\begin{frame}

	\figh{\string~/Dropbox/statistics/work/vandy/AdaptiveMonitoring/manuscript/figs/probEffLooks20}{.9}

\end{frame}


\begin{frame}

	\figh{\string~/Dropbox/statistics/work/vandy/AdaptiveMonitoring/manuscript/figs/averageNLooks20}{.9}

\end{frame}




\begin{frame}[c]

	\figh{\string~/Dropbox/statistics/work/vandy/AdaptiveMonitoring/manuscript/figs/biasLooks20}{.9}

\end{frame}



\begin{frame}[c]

	\figh{\string~/Dropbox/statistics/work/vandy/AdaptiveMonitoring/manuscript/figs/coverageLooks20}{.9}

\end{frame}


\begin{frame}

Paper 3: sgpvAM R package and practical guidance to control adaptive monitoring errors

\end{frame}




\begin{frame}{sgpvAM package}

1. How to easily operationalize?

2. Encourage not changing clinical relevance

3. Under fully sequentially monitored study, how long to wait to ensure Type I Error < 0.05?

4. What are operating characteristics when observations are not observed immediately?  When limited in sample size?

\end{frame}


\begin{frame}{sgpvAM package}
sgpvAM package

1. Uses Valerie's spgv package on github

2. Vignette with examples

-- Example 1-sided study with at most trivial effects = 0.2 and minimal highly actionable effect of 0.5, lag of 50 observations, will monitor no more frequently than 10 patients.

\end{frame}


\begin{frame}{sgpvAM package}

\figw{\string~/Dropbox/statistics/work/vandy/AdaptiveMonitoring/manuscript/figs/sgpvAMexampleRejPN}{.9}

\end{frame}


\begin{frame}{sgpvAM package}

\figw{\string~/Dropbox/statistics/work/vandy/AdaptiveMonitoring/manuscript/figs/sgpvAMexampleAveN}{.9}

\end{frame}


\begin{frame}{sgpvAM package}

\figw{\string~/Dropbox/statistics/work/vandy/AdaptiveMonitoring/manuscript/figs/sgpvAMexampleProbMore}{.9}

\end{frame}

\begin{frame}{sgpvAM package}

\figh{\string~/Dropbox/statistics/work/vandy/AdaptiveMonitoring/manuscript/figs/waitTimeRejPN}{.8}

\end{frame}

\begin{frame}{sgpvAM package}

\figh{\string~/Dropbox/statistics/work/vandy/AdaptiveMonitoring/manuscript/figs/waitTimeInconclusiveThetaDeltaMidPoint}{.8}

\end{frame}



% \begin{frame}{Acknowledgements}
%
% Joint work with:
% \bi
%   \item Robert A. Greevy, Jr, PhD
%   \item Lindsay Mayberry, PhD
% \ei
%
% \vspace{0.3cm}
% REACH RO1 Funding: NIDDK R01 DK100694
%
% \end{frame}

\bibliographystyle{plainnat}
\begin{frame}{References}
  \addtocounter{framenumber}{-1}
  \bibliography{citations}
\end{frame}
%
%
% \begin{frame}{End of Study Treatment Balance}
% \addtocounter{framenumber}{-1}
%
% \figw{\string~/Dropbox/statistics/work/vandy/RandomizationMethods/ExtendingMOTF/FigsRefs/visualPresentations/trtBalance}{.7}
% \end{frame}
%
% % \section{Additional Results}
%
%
% \begin{frame}[fragile]{ }
% \addtocounter{framenumber}{-1}
% \figw{\string~/Dropbox/statistics/work/vandy/RandomizationMethods/ExtendingMOTF/FigsRefs/visualPresentations/MaxSMD_uu_bonus}{.825}
% \end{frame}
%
%
% \begin{frame}[fragile]{ }
% \addtocounter{framenumber}{-1}
% \figw{\string~/Dropbox/statistics/work/vandy/RandomizationMethods/ExtendingMOTF/FigsRefs/visualPresentations/AveSMD_uu_bonus}{.825}
% \end{frame}
%
%
% \begin{frame}[fragile]{ }
% \addtocounter{framenumber}{-1}
% \figw{\string~/Dropbox/statistics/work/vandy/RandomizationMethods/ExtendingMOTF/FigsRefs/visualPresentations/modelAdjCIwidth_uu_bonus}{.825}
% \end{frame}
%
% \begin{frame}[fragile]{ }
% \addtocounter{framenumber}{-1}
%
% \figw{\string~/Dropbox/statistics/work/vandy/RandomizationMethods/ExtendingMOTF/FigsRefs/visualPresentations/permutationCIwidth_uu2_bonus}{.825}
%
% \end{frame}
%
%
% \begin{frame}
% \addtocounter{framenumber}{-1}
%
% \figw{\string~/Dropbox/statistics/work/vandy/RandomizationMethods/ExtendingMOTF/FigsRefs/visualPresentations/resSize}{0.8}
%
% \end{frame}



\end{document}
